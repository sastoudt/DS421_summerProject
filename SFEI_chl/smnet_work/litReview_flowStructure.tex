
\documentclass[12pt]{amsart}
\usepackage{geometry} % see geometry.pdf on how to lay out the page. There's lots.
\geometry{a4paper} % or letter or a5paper or ... etc
\usepackage{color}
% \geometry{landscape} % rotated page geometry

% See the ``Article customise'' template for come common customisations

\title{Informal Lit Review for Flow Structure}
\author{Sara Stoudt}
%\date{} % delete this line to display the current date

%%% BEGIN DOCUMENT
\begin{document}

\maketitle
%\tableofcontents

\section{A mixed-model moving-average approach to geostatistical modeling
in stream networks, Peterson, Ver Hoeff, Ecology 2010}

(more of an overview piece)

Problems with Euclidean Distance:  does not represent 
\begin{itemize}
\item spatial configuration
\item longitudinal connectivity: "refers to the pathway along the entire length of a stream. A gradient of physical, chemical, and biological processes occur from source to mouth."
\item discharge
\item flow direction
\item common autocovariance functions are not generally valid when Euclidean distance is replaced with a hydrologic distance measure
\item generally valid autocovariance function must be guaranteed to produce a covariance matrix that is symmetric and positive definite with all nonnegative diagonal elements regardless of the configuration of your data
\end{itemize}

Hydrologic distance
\begin{itemize}
\item distance along the flow network
\item ``flow-connected" relationship requires water flow from one location to another for two sites to be correlated
\item ``flow-unconnected": sites are spatially independent if no flow between the two
\end{itemize}

Tail-up models
\begin{itemize}
\item use hydrologic distance, only allow autocorrelation for flow-connected relationships
\item autocorrelation occurs when moving average functions overlap among sites, with greater autocorrelation resulting from greater overlap
\item when the moving average (MA) function reaches a confluence (split) in the network, segment weights are used to proportionally allocate or split the function between upstream segments (often use discharge or watershed area to represent relative influence in a stream network)
\item segment weights can also be based on measures that represent the sum of the upstream measures (segment does not contriubte anything to itself) such as Shreve's stream order [basically how many stream segments feed into that one]
\item segment weights that sum to one ensure constant variance
\item good for modeling organisms or materials that move passively downstream

\end{itemize}

Tail-down models
\begin{itemize}
\item allow autocrrelation between both flow-connected and flow-unconnected pairs of sites
\item MA function is only non-zero downstream of a location
\item total hydrologic distance is used for flow-connected pairs, but the hydrologic distances from each site to a common confluence are used for flow-unconnected pairs
\item no need for segment weights since MA function points downstream
\item restriction for relative strength of spatial autocorrelation for each type [flow connected and unconnected]
\begin{itemize}
\item one pair is flow-connected, one pair is not, total distance for flow connected = sum of distances of non-flow connected points to a shared confluence, the strength of the spatial autocorrelation is generally equal or greater for the flow-unconnected pairs
\item none of the current models are able to generate a tail-down (or up) model with significantly stronger spatial autocorrelation between flow-connected pairs than flow-unconnected pairs for an equal hydrologic distance
\item restrictions might make sense for fish populations that have the tendency to invade upstream reaches
\end{itemize}
\end{itemize}

Mixed models
\begin{itemize}
\item variance component approach
\item Covariance mixture = $\sigma^2_{Euc}cor(z_{Euc})+\sigma^2_{TD}cor(z_{TD})+\sigma^2_{TU}cor(z_{TU})+\sigma^2_{nug}I$
\item $cor(z_{TU})= C_{TU} \odot W_{TU}$ elementwise multiplication of flow-connected autocorrelations and the spatial weights matrix 
\item allows for the possibility of more autocorrelation among flow-connected pairs of sites with somewhat less autocorrelation among flow-unconnected pairs of sites
\item multiple range parameters can capture patterns at multiple scales
\end{itemize}

In practice
\begin{itemize}
\item recommend fitting a full covariance matrix, allows the data to determine which variance components have the strongest influence
\item TU model allows discontinuities in predictions at confluences
\item effect of spatial weights on prediction uncertainty is apparent if upstream segments have not been sampled (various combinations can contribute to the downstream observed)
\item possible to produce covariance matrices for Poisson or binomial variables
\end{itemize}

\section{Geostatistical modelling on stream networks: developing
valid covariance matrices based on hydrologic distance
and stream flow, Peterson, Theobald, Ver Hoef, Freshwater Biology 2007}

(more of a detailed paper)

buzz words
\begin{itemize}
\item hierarchy theory (http://www.botany.wisc.edu/allenlab/AllenLab/Hierarchy.html)
\item the need to recognize multi-scale processes
\end{itemize}

Distance measures
\begin{itemize}
\item symmetric distance: allows movement between sites in all directions (Euclidean)
\item symmetric hydrologic distance: shortest distance along stream network without restricting to flow
\item asymmetric hydrologic distance: requires water flow from one location to another for two sites to be connected (upstream or downstream, not both)
\end{itemize}

Geostatistical modelling in stream networks
\begin{itemize}
\item nugget: represents variation between sites as their separation distance approaches zero, experimental error, or indicate that substantial variation occurs at a scale finer than the sampling scale
\item sill: autocovariance asymptote, represents variance found among uncorrelated data
\item range: how fast the autocovariance decays with distance
\item covariance matrix must by symmetric, positive-definite, and ll diagonal elemnts must be non-negative
\item exponential model is valid when making predictions at unobserved locations using covariance matrices based on symmetric hydrologic distance
\item pure asymmetric distance measures (unweighted) do not produce symmetric covariance matrices
\end{itemize}

Autocovariance models using hydrologic distances
\begin{itemize}
\item large class of autocovariance functions can be developed by creating random variables as the integration of a moving average function over a white noise random process
\item $Z(s) \int_{-\infty}^\infty g(x-s|\theta)W(x)dx$
\item $C(h|\theta)=  \int_{-\infty}^\infty (g(x|\theta))^2dx+\theta_0$ (if h=0) $\int_{-\infty}^\infty g(x|\theta)g(x-h|\theta)dx$ (if h>0)
\item $\theta_0$ is nugget, $h$ is the separation distance
\item distance upstream: distance upstream of any segment from the stream outlet
\end{itemize}

Notation
\begin{itemize}
\item $x_i$: distance upstream on the $i$th stream segment
\item $l_i$: most downstream location on the $i$th stream segment
\item $u_i$: most upstream location on the $i$th stream segment
\item $l_i=u_j$ when the $i$th segment is directly upstream from the $j$th segment
\item $U_{x_i}$: index set of stream segments upstream of $x_i$ excluding $i$
\item $B_{x_i, s_j}$: index set of segments between downstream location $x_i$ and upstream location $s_j$ excluding the downstream segment but including the upstream segment
\item $B_{x_i,[j]}$: index set between a downstream location and an upstream segment $j$
\item $\omega_k$: segment weight equal to the proportion that a stream segment contributes to the segment directly downstream
\item $Z(s_i)=\int_{s_i}^{u_i}g(x_i-s_i|\theta)W(x_i)dx_i + \sum_{j \in U_{s_i}} \left( \prod_{k \in B_{s_i,[j]}}\sqrt{\omega_k}\right) \int_{l_j}^{u_j} g(x_j-s_i|\theta)W(x_j)dx_j$
\item in words: moving average from point of interest to upstream of moving average function over the white noise process plus sum of [(product of weights from those in-between location of interest and an upstream segment)*  (lower to upper of stream segment j moving average over random process)] for all segments upstream of location of interest
\item $C(s_i,s_j|\theta) = 0$ (if locations are not flow connected) $= C_1(0)_\theta_0$ (if location 1=location 2) $= \prod_{k \in B_{s_i,s_j} \sqrt{\omega_k}C_1(|s_i-s_j|)}$ (otherwise)
\item $C_1(h)= \int_{-\infty}^\infty g(x|\theta) g(x-h|\theta)dx$
\item exponential, linear with sill, spherical, or Mariah autocovariance functions can be fit to the empirical covariances once the asymmetric hydrologic distance data have been weighted appropriately
\item can plug covariance matrix generated from this autocovariance function into kriging
\end{itemize}

In practice
\begin{itemize}
\item need to calculate symmetric and asymmetric hydrologic distance measures using ARCGIS
\item for spatial weights: need to calculate discharge volume or use watershed area as a surrogate
\item V: uses the hydrologic distance between each pair of locations $h$, $C_1(h|\theta)= \theta_0+\theta_1$ (if h=0), $\theta_1 \rho(h/\theta_2)$ (if h>0)
\item then elementwise product $V$ with a symmetric spatial weights matrix W+W'
\end{itemize}

Future directions
\begin{itemize}
\item create other relevant distance measures that incorporate physical characteristics such as flow velocity, stream gradient, or physical structures that better reflect the "energy" an organism expends to move from one location to another
\item network connectivity could also include chemical, physical, and biological barriers
\end{itemize}

\section{Flexible regression models over river networks, O' Donnel, Rushworth, Bowman, and Scott, Applied Statistics 2014}

Overview of what the goal is
\begin{itemize}
\item underlying trends that are more flexible than linear trends
\item smoothing techniques not seen often in a network setting
\item borrow strength locally while respecting the specific topology of a network and the additional complications of directionality and size of flow
\item how to deal with confluence points where different branches of the network combine
\item want to be allowed to predict sharp changes which are often expected at confluence points
\item extend to spatial, temporal, covariate effects, spatio-temporal
\end{itemize}

Kernel Functions
\begin{itemize}
\item local average $\hat{m}(x)=\frac{\sum_i w(x_i-x; h) y_i}{\sum_i w(x_i-x;h)}$
where the weight function $w$ decreases with distance from zero at a rate determined by $h$ ($w$ could be normal density and $h$ could be the standard deviation)
\item can use river distance and make weights non-zero only where there is a flow path between two points, additional weights can reflect the volume of water flowing in different sections of the river
\item can fit Ver Hoef approach into this framework
\end{itemize}

Penalized Splines
\begin{itemize}
\item $\hat{m}(x)=\sum_j \beta_j \phi_j(x)$ where the $\phi$s are basis functions
\item dimensionality can be kept low, independent of the size of the data
\item B splines, or penalized B splines for computational expediency
\item construct a basis set over a network, need to combine them suitably at the confluence points
\begin{itemize}
\item smooth overlapping functions
\item divide network into a large number of small pieces within which the function $m$ is likely to change very little, piecewise constant (B splines of order 0)
\end{itemize}
\item need to penalize because more stream units than observed values, smoothness of $\beta$ values corresponding to adjacent stream units $j$ and $k$ with no intervening confluence can be measured by $(\beta_j-\beta_k)^2$
\item if there is a confluence point (a and b feed into c) then $f_c=f_a+f_b$, relative flows $\omega_a=f_a/f_c$ and $\omega_b=f_b\f_c$
\item $\omega_a\beta_a+\omega_b\beta_b=\beta_c$, so we need \omega_a(\beta_a-\beta_c)+\omega_b(\beta_b-\beta_c)=0$
\item use this difference penalty: $\lambda\{\omega_a^2(\beta_a-\beta_c)^2+\omega_b^2(\beta_b-\beta_c)^2 \}$
\item outputs conditioned on their immediate upstream neighbors are independent of all further upstream neighbors
\item P-spline model: $y=B\beta+\epsilon$, $B$ is an $n \times p$ indicator matrix whose $i$th row has the value 1 in the column corresponding to the stream unit of $y_i$ and 0s elsewhere
\item minimize the penalized sum of squares: $(y-B\beta)^T(y-B\beta)+\lambda\beta^TD^TD\beta$ with respect to $\beta$
\item $D$: generates the differences between $\beta$s from adjacent stream units, weighted by flow when there is an intervening confluence point, $D$ is $p\times p$, $p$ is number of stream pieces
\item $\hat{\beta}=(B^TB+\lambda D^TD)^{-1}B^Ty$
\end{itemize}

Computational Comparisons
\begin{itemize}
\item easy to identify estimated values at any location of interest by identifying the stream unit in which it lies
\item local mean estimation requires a new calculation for each new estimation point
\item the penalty can bridge the gap between data points if there are sparse regions
\item local mean estimation can run into difficulties with very small weights
\item P-spline construction only deals with relationship between values and flows in neighboring stream units, so you don't need to define connectedness across the whole network simultaneously or to consider the cumulative effect of flow across distances which span several confluences
\item can lose some information because the lengths of the stream units are not used, small if stream units are defined as homogeneous stretches of water
\item P splines will scale up better
\item have to choose $\lambda$ for P splines
\end{itemize}

Spatiotemporal
\begin{itemize}
\item spatial, temporal, and seasonal trends: $y_i=\mu+m_s(s_i)+m_t(t_i)+m_z(z_i)+\epsilon_i$
\item can use cubic B-splines for the temporal and seasonal effects, defined over more standard sample spaces, can combine all of the design matrices for each piece into one
\item smoothness penalty: $\beta^TP\beta$ where $P$ has block diagonal form that combines the individual penalties $(0, \lambda_sD_s^TD_s, \lambda_tD_t^TD_t, \lambda_zD_z^TD_z)$
\item $D_t$ often a second order differencing, $D_z$ needs a cyclic structure so that the first $r$ basis functions are identical with the last $r$: $\sum_{k=1}^r (\beta_{z,k}-\beta_{z,p+1-k})^2$
\item for identifiability, need a ridge penalty: $\beta^TQ\beta$ where $Q$ is a diagonal matrix constructed from $(0, \nu_s 1_s, \nu_t1_t, \nu_z1_z)$ where the $\nu$ parameters are the ridge  parameters, $1_s$ a vector of 1s whose length is determined by the basis functions in the $s$ term
\item $\hat{\beta} =(B^TB+P+Q)^{-1}B^Ty$
\item can get standard errors for $\hat{\beta}$ from the diagonal elements of $HH^T$ $(H=(B^TB+P+Q)^{-1}B^T) $ multiplied by an estimate of the error variance: $$\hat{\sigma}^2$=RSS/(n-dof) [residual sum of squares and approximate degrees of freedom for the model (trace of hat matrix)]
\end{itemize}

Allow for interaction terms
\begin{itemize}
\item let the spatial pattern evolve over time
\item $y_i=\mu+m_s(s_i)+m_t(t_i)+m_z(z_i)+m_{s,t}(s_i,t_i)+m_{s,z}(s_i,z_i)+m_{t,z}(t_i,z_i)+\epsilon_i$
\item basis formed by all possible products of the spline basis functions on each separate variable
\item $m_{s,t}=\sum_{j,k} \beta_{jk}\phi_{s,j}\phi_{t,k}$
\item $B=(1 B_s B_t B_z B_x \square B_t B_s\square B_z B_t \square B_z)$
\item $\square$ is the row-wise tensor product: $A\squareY=(A\otimes 1') \odot (1' \otimes Y) $ where $\odot$ is elementwise prod \textcolor{red}{look into dimensions of this}
\item penalties for the interaction terms can be constructed by considering the coefficients $\beta_{jk}$ in matrix form and applying smoothness penalties to both the rows and columns: (ex.) space-time smoothness, apply first-order network penalty to the columns and a second-order difference penalty over the rows
\end{itemize}

Computational details
\begin{itemize}
\item tensor product spline smooths have many basis parameters to fit
\item $V$ denotes the $n \times q$ matrix evaluating the bases for the other terms in the model, then $B_s \square V$ is at least $100(1-1/p_s)\%$ sparse and much more so if $V$ is sparse
\item define all of the model components in a compressed format, spam or Matrix packages)
\item Cholesky factor $L$ of $B^TB+P+Q$ to obtain the quadratic form of interest where $B^TB+P+Q=LL^T$
\item standard errors of the fitted values are the diagonal elements of $B^TL^{-1T}L^{-1}BB^TL^{-T}L^{-1}B$ and can be calculated efficiently by solving first $AL=B$ then $CL^T=A$ and then summing the elementwise product of $CB^T$ with itself
\end{itemize}

Residual correlation
\begin{itemize}
\item residual temporal correlation at short time lags is to be expected, the model accounts for trends over longer time periods
\item standard error estimates are likely to be underestimated when the underlying error process is correlated
\item conservative approach: fit a separable spatiotemporal model to the errors so that $\hat{\Sigma){ij}=cov(\epsilon_i,\epsilon_j)=\omega_{ij}\sigma^2 \exp{-d_{ij}/\rho-|t_i-t_j|/\psi}}$
\item $\omega_{ij}=\prod_{k\in N} \omega_k$, k indexes teh set of stream units that lie between $i$ and $j$ and on the same flow path as both
\item time lag: t_i-t_j, d_{ij}: network separation measured in numbers of stream units
\item fit by weighted least squares
\item adjustment: $se(\hat{y})=\sqrt{var(\hat{H}y}=\sqrt{diag(\hat{H}\hat{\Sigma}\hat{H}^T}$, \hat{H}=B(B^TB+P+Q)^{-1}B^T
\item could incorporate the correlation structure into the fitting process but would increase the complexity of the computations because sparsity would be compromised
\end{itemize}

Future work
\begin{itemize}

\item smoothing parameters representing variance components which can be estimated by restricted maximum likelihood
\item Bayesian hierarchical model interpretation: $\beta_i$ are treated as random effects with a Gaussian Markov random-field prior and a fixed variance, uncertainty associated with smoothing parameters is integrated out (right now a grid search)
\item need average flow values for each stream unit in the network partition, flow ratios are the crucial quantities
\end{itemize}


\end{document}
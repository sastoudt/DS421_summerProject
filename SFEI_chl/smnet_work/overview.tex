
\documentclass[12pt]{amsart}
\usepackage{geometry} % see geometry.pdf on how to lay out the page. There's lots.
\geometry{a4paper} % or letter or a5paper or ... etc
% \geometry{landscape} % rotated page geometry
\usepackage{float}
% See the ``Article customise'' template for come common customisations

\title{Overview of Leveraging smnet for Our Problem}
\author{Sara Stoudt}
%\date{} % delete this line to display the current date

%%% BEGIN DOCUMENT
\begin{document}

\maketitle
%\tableofcontents

smnet is an R package based on the paper: ``Flexible regression models over river networks" by O' Donnell, Rushworth, Bowman, and Scott (2014). litReview\_flowStructure.tex contains an overview of this method. The main hurdle to testing out this method in this situation is that the main function relies on the data being in a format that requires extensive preprocessing in GIS. I was able to more or less work around this and reproduce the main functionality of this approach using the source code of the package, building the network structure manually, and massaging the data into a format that could reasonable be propagated forward throughout the method.

I removed C10 and D41 from the network to simplify the network structure. I chose D6 to be the sink. This was somewhat an arbitrary choice between D6 and D7. I also fudged the system so that D7 and D8 ``flow into" D6 which is not really true, since I needed one sink. Below, I have included the spatial information needed to determine the spatial penalty matrix (shreve order, binary ID, and adjacency matrix). This information is usually obtained through GIS routines. Note: the spatial penalty matrix $D$ is $13 \times 13$.

\begin{table}[H]
\centering
\begin{tabular}{lll}
Station & Shreve Order & Binary ID \\
C3      & 1            & 111011    \\
D10     & 4            & 111       \\
D12     & 3            & 1111      \\
D19     & 3            & 11111     \\
D22     & 1            & 11101     \\
D26     & 2            & 111110    \\
D28A    & 1            & 111111    \\
D4      & 1            & 1110      \\
D6      & 5            & 1         \\
D7      & 1            & 10        \\
D8      & 4            & 11        \\
MD10    & 1            & 1111100   \\
P8      & 1            & 1111101  
\end{tabular}
\caption{Manual Network Information}
\label{tab:networkInfo}
\end{table}

\begin{verbatim}

sfeiAdjMatrix<-matrix(0,nrow=13,ncol=13)

sfeiAdjMatrix[1,5]=1 ## C3 --> D22

sfeiAdjMatrix[2,11]=1 ## D10 --> D8

sfeiAdjMatrix[3,2]=1 ## D12 --> D10

sfeiAdjMatrix[4,3]=1 ## D19 --> D12

sfeiAdjMatrix[5,8]=1 ## D22 --> D4

sfeiAdjMatrix[6,4]=1 ## D26 --> D19

sfeiAdjMatrix[7,4]=1 ## D28A --> D19

sfeiAdjMatrix[8,2]=1 ## D4 --> D10

sfeiAdjMatrix[11,9]=1 ## D8 --> D6

sfeiAdjMatrix[11,10]=1 ## D8 --> D7

sfeiAdjMatrix[12,6]=1 ## MD10 --> D28A

sfeiAdjMatrix[13,6]=1 ## P8 --> D28A
\end{verbatim}

Since in our case we are only interested in predicting values at the stations, not other points in the stream network, each station is on its own ``stream segment" instead of breaking the system into many more small ``stream segments". 

For temporal smooths of both day of year (cyclical: cSplineDes) and date trend (cubic regression splines: splineDesign), I use already established functions in R to generate the model matrices.

The spatial penalty matrix needs weights to combine penalties for smoothness across each flow path of the confluences. These weights are supposed to be determined by relative flow volumes, but there is an option to use shreve weights. Unfortunately, I have not quite gotten the hang of choosing these weights. My shreve values do not pass the weight check implemented by smnet. Since I move forward, reproducing the functionality, I can go ahead and use these weights anyway, but they are somehow not optimal. I am still trying to understand the necessity of the weight check and why my version of shreve order does not pass.

I tried other weights: normalized shreve order, weights $w$ as described in the O'Donnell paper, all zero, and all one. There is only a slight amount of variation in the overall RMSE with these different weights, which is puzzling. This could be because the smoothing parameter is so small, so any differences are negligible on that scale, but I still have a hard time buying that the spatial weights don't impact the overall performance. I am especially confused why the all zero weights scenario doesn't make a difference (and actually has the smallest RMSE, albeit marginally), since this should essentially boil down to knowing what station each point is at, but having no information about the flow structure.

The smoothness parameters and the ridge parameters must also be chosen. The smnet functions optimize these, so I try to leverage that work in my deconstructed implementation. I pulled out the relevant optimization procedure that uses Generalized Cross Validation to optimize the smoothing parameters ($\lambda$s). I then manually did five fold cross validation to optimize the ridge parameters ($\nu$s). Overall, the RMSE seems robust to the choice of the ridge parameter within a reasonable range. 

Breaking down the RMSEs by station, there isn't a major change from which stations are ``hard" when compared to results from other model attempts. The RMSEs per station are pretty variable and overall bigger than the RMSEs of our more successful models. 

Breaking down the RMSEs by time, the RMSEs get better over time and are generally better at the beginning and the end of the year. The overall fitted values are not nearly extreme enough but the predictions generally seem to be in phase with the true values.

\textit{Trying to Understand Spatial Weights Matrix D}

I visualized the spatial weights matrix by flattening it, color coding the weights, and plotting on top of the actual flow map. When I do this, I find different properties of the different weighting schemes, so the different weights \textbf{are} defining different structures. This just does not seem to propagate forward into the predictions.


For the normalized shreve (and regular shreve since proportional), the upper branch segments have weights that are the smallest in magnitude weights. Weights based off of the O'Donnell approach for the upper branch have weights that are highest in magnitude.  The left most segments (end of flow) are the highest in magnitude for the shreve weights while the magnitudes of the segments in this region are more in the middle of the weight spectrum. There is a big change in the weight magnitudes between D10, D4, and D22 and between D12 and D19 on the weights based on the O'Donnell approach. The transitions between weights are smoother for the shreve weights.

Even though the overall RMSEs don't differ very much between the different spatial penalty matrices, perhaps the break-down by station differs. There is not much variability between the RMSEs by station either. There does seem to be some evidence that stations at higher shreve orders have higher variability in their RMSEs. This suggests that the penalty matrices differ most for more periphery stations.

If we just set up the temporal smooths in the O'Donnell framework without the spatial penalty matrix $D$ we get an overall RMSE that is only 0.21 larger than when we include $D$. However, we should note that this isn't a completely fair comparison since we use the ``optimal" smoothing parameters from the case derived with the $D$ matrix which may no longer be optimal for this case. Even with this caveat, it seems like the spatial penalty is not actually helping us very much in this case. When we break down the RMSE by station, the RMSEs are a little worse for the rightmost stations and the cluster of three stations close to the right than the RMSEs from other models. When we break down the RMSEs over time and seasons, we see the same patterns as when we included $D$.





\textbf{}

\end{document}
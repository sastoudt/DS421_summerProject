
\documentclass[12pt]{amsart}
\usepackage{geometry} % see geometry.pdf on how to lay out the page. There's lots.
\geometry{a4paper} % or letter or a5paper or ... etc
% \geometry{landscape} % rotated page geometry

% See the ``Article customise'' template for come common customisations

\title{Informal Lit Review for Flow Structure}
\author{Sara Stoudt}
%\date{} % delete this line to display the current date

%%% BEGIN DOCUMENT
\begin{document}

\maketitle
%\tableofcontents

\section{A mixed-model moving-average approach to geostatistical modeling
in stream networks, Peterson, Ver Hoeff, Ecology 2010}

(more of an overview piece)

Problems with Euclidean Distance:  does not represent 
\begin{itemize}
\item spatial configuration
\item longitudinal connectivity: "refers to the pathway along the entire length of a stream. A gradient of physical, chemical, and biological processes occur from source to mouth."
\item discharge
\ite flow direction
\item common autocovariance functions are not generally valid when Euclidean distance is replaced with a hydrologic distance measure
\item generally valid autocovariance function must be guaranteed to produce a covariance matrix that is symmetric and positive definite with all nonnegative diagonal elements regardless of the configuration of your data
\end{itemize}

Hydrologic distance
\begin{itemize}
\item distance along the flow network
\item ``flow-connected" relationship requires water flow from one location to another for two sites to be correlated
\item ``flow-unconnected": sites are spatially independent if no flow between the two
\end{itemize}

Tail-up models
\begin{itemize}
\item use hydrologic distance, only allow autocorrelation for flow-connected relationships
\item autocorrelation occurs when moving average functions overlap among sites, with greater autocorrelation resulting from greater overlap
\item when the moving average (MA) function reaches a confluence (split) in the network, segment weights are used to proportionally allocate or split the function between upstream segments (often use discharge or watershed area to represent relative influence in a stream network)
\item segment weights can also be based on measures that represent the sum of the upstream measures (segment does not contriubte anything to itself) such as Shreve's stream order [basically how many stream segments feed into that one]
\item segment weights that sum to one ensure constant variance
\item good for modeling organisms or materials that move passively downstream

\end{itemize}

Tail-down models
\begin{itemize}
\item allow autocrrelation between both flow-connected and flow-unconnected pairs of sites
\item MA function is only non-zero downstream of a location
\item total hydrologic distance is used for flow-connected pairs, but the hydrologic distances from each site to a common confluence are used for flow-unconnected pairs
\item no need for segment weights since MA function points downstream
\item restriction for relative strength of spatial autocorrelation for each type [flow connected and unconnected]
\begin{itemize}
\item one pair is flow-connected, one pair is not, total distance for flow connected = sum of distances of non-flow connected points to a shared confluence, the strength of the spatial autocorrelation is generally equal or greater for the flow-unconnected pairs
\item none of the current models are able to generate a tail-down (or up) model with significantly stronger spatial autocorrelation between flow-connected pairs than flow-unconnected pairs for an equal hydrologic distance
\item restrictions might make sense for fish populations that have the tendency to invade upstream reaches
\end{itemize}
\end{itemize}

Mixed models
\begin{itemize}
\item variance component approach
\item Covariance mixture = $\sigma^2_{Euc}cor(z_{Euc})+\sigma^2_{TD}cor(z_{TD})+\sigma^2_{TU}cor(z_{TU})+\sigma^2_{nug}I$
\item $cor(z_{TU})= C_{TU} \odot W_{TU}$ elementwise multiplication of flow-connected autocorrelations and the spatial weights matrix 
\item allows for the possibility of more autocorrelation among flow-connected pairs of sites with somewhat less autocorrelation among flow-unconnected pairs of sites
\item multiple range parameters can capture patterns at multiple scales
\end{itemize}

In practice
\begin{itemize}
\item recommend fitting a full covariance matrix, allows the data to determine which variance components have the strongest influence
\item TU model allows discontinuities in predictions at confluences
\item effect of spatial weights on prediction uncertainty is apparent if upstream segments have not been sampled (various combinations can contribute to the downstream observed)
\item possible to produce covariance matrices for Poisson or binomial variables
\end{itemize}

\subsection{Geostatistical modelling on stream networks: developing
valid covariance matrices based on hydrologic distance
and stream flow, Peterson, Theobald, Ver Hoef, Freshwater Biology 2007}

(more of a detailed paper)

buzz words
\begin{itemize}
\item hierarchy theory (http://www.botany.wisc.edu/allenlab/AllenLab/Hierarchy.html)
\item the need to recognize multi-scale processes
\end{itemize}

Distance measures
\begin{itemize}
\item symmetric distance: allows movement between sites in all directions (Euclidean)
\item symmetric hydrologic distance: shortest distance along stream network without restricting to flow
\item asymmetric hydrologic distance: requires water flow from one location to another for two sites to be connected (upstream or downstream, not both)
\end{itemize}

Geostatistical modelling in stream networks
\begin{itemize}
\item nugget: represents variation between sites as their separation distance approaches zero, experimental error, or indicate that substantial variation occurs at a scale finer than the sampling scale
\item sill: autocovariance asymptote, represents variance found among uncorrelated data
\item range: how fast the autocovariance decays with distance
\item covariance matrix must by symmetric, positive-definite, and ll diagonal elemnts must be non-negative
\item exponential model is valid when making predictions at unobserved locations using covariance matrices based on symmetric hydrologic distance
\item pure asymmetric distance measures (unweighted) do not produce symmetric covariance matrices
\end{itemize}

Autocovariance models using hydrologic distances
\begin{itemize}
\item large class of autocovariance functions can be developed by creating random variables as the integration of a moving average function over a white noise random process
\item $Z(s) \int_{-\infty}^\infty g(x-s|\theta)W(x)dx$
\item $C(h|\theta)=  \int_{-\infty}^\infty (g(x|\theta))^2dx+\theta_0$ (if h=0) $\int_{-\infty}^\infty g(x|\theta)g(x-h|\theta)dx$ (if h>0)
\item $\theta_0$ is nugget, $h$ is the separation distance
\item distance upstream: distance upstream of any segment from the stream outlet
\end{itemize}

Notation
\begin{itemize}
\item $x_i$: distance upstream on the $i$th stream segment
\item $l_i$: most downstream location on the $i$th stream segment
\item $u_i$: most upstream location on the $i$th stream segment
\item $l_i=u_j$ when the $i$th segment is directly upstream from the $j$th segment
\item $U_{x_i}$: index set of stream segments upstream of $x_i$ excluding $i$
\item $B_{x_i, s_j}$: index set of segments between downstream location $x_i$ and upstream location $s_j$ excluding the downstream segment but including the upstream segment
\item $B_{x_i,[j]}$: index set between a downstream location and an upstream segment $j$
\item $\omgega_k$: segment weight equal to the proportion that a stream segment contributes to the segment directly downstream
\item $Z(s_i)=\int_{s_i}^{u_i}g(x_i-s_i|\theta)W(x_i)dx_i + \sum_{j \in U_{s_i}} \left( \prod_{k \in B_{s_i,[j]}}\sqrt{\omega_k}\right) \int_{l_j}^{u_j} g(x_j-s_i|\theta)W(x_j)dx_j$
\item in words: moving average from point of interest to upstream of moving average function over the white noise process plus sum of [(product of weights from those in-between location of interest and an upstream segment)*  (lower to upper of stream segment j moving average over random process)] for all segments upstream of location of interest
\item $C(s_i,s_j|\theta) = 0$ (if locations are not flow connected) $= C_1(0)_\theta_0$ (if location 1=location 2) $= \prod_{k \in B_{s_i,s_j} \sqrt{\omega_k}C_1(|s_i-s_j|)}$ (otherwise)
\item C_1(h)= \int_{-\infty}^\infty g(x|\theta) g(x-h|\theta)dx$
\item exponential, linear with sill, spherical, or Mariah autocovariance functions can be fit to the empirical covariances once the asymmetric hydrologic distance data have been weighted appropriately
\item can plug covariance matrix generated from this autocovariance function into kriging
\end{itemize}

In practice
\begin{itemize}
\item need to calculate symmetric and asymmetric hydrologic distance measures using ARCGIS
\item for spatial weights: need to calculate discharge volume or use watershed area as a surrogate
\item V: uses the hydrologic distance between each pair of locations $h$, $C_1(h|\theta)= \theta_0+\theta_1$ (if h=0), $\theta_1 \rho(h/\theta_2)$ (if h>0)
\item then elementwise product $V$ with a symmetric spatial weights matrix W+W'
\end{itemize}

Future directions
\begin{itemize}
\item create other relevant distance measures that incorporate physical characteristics such as flow velocity, stream gradient, or physical structures that better reflect the "energy" an organism expends to move from one location to another
\item network connectivity could also include chemical, physical, and biological barriers
\end{itemize}

\end{document}

\documentclass[12pt]{amsart}
\usepackage{geometry} % see geometry.pdf on how to lay out the page. There's lots.
\geometry{a4paper} % or letter or a5paper or ... etc
% \geometry{landscape} % rotated page geometry

% See the ``Article customise'' template for come common customisations

\title{Informal Lit Review for Flow Structure}
\author{Sara Stoudt}
%\date{} % delete this line to display the current date

%%% BEGIN DOCUMENT
\begin{document}

\maketitle
%\tableofcontents

\section{A mixed-model moving-average approach to geostatistical modeling
in stream networks, Peterson, Ver Hoeff, Ecology 2010}

Problems with Euclidean Distance:  does not represent 
\begin{itemize}
\item spatial configuration
\item longitudinal connectivity: "refers to the pathway along the entire length of a stream. A gradient of physical, chemical, and biological processes occur from source to mouth."
\item discharge
\ite flow direction
\item common autocovariance functions are not generally valid when Euclidean distance is replaced with a hydrologic distance measure
\item generally valid autocovariance function must be guaranteed to produce a covariance matrix that is symmetric and positive definite with all nonnegative diagonal elements regardless of the configuration of your data
\end{itemize}

Hydrologic distance
\begin{itemize}
\item distance along the flow network
\item ``flow-connected" relationship requires water flow from one location to another for two sites to be correlated
\item ``flow-unconnected": sites are spatially independent if no flow between the two
\end{itemize}

Tail-up models
\begin{itemize}
\item use hydrologic distance, only allow autocorrelation for flow-connected relationships
\item autocorrelation occurs when moving average functions overlap among sites, with greater autocorrelation resulting from greater overlap
\item when the moving average (MA) function reaches a confluence (split) in the network, segment weights are used to proportionally allocate or split the function between upstream segments (often use discharge or watershed area to represent relative influence in a stream network)
\item segment weights can also be based on measures that represent the sum of the upstream measures (segment does not contriubte anything to itself) such as Shreve's stream order [basically how many stream segments feed into that one]
\item segment weights that sum to one ensure constant variance
\item good for modeling organisms or materials that move passively downstream

\end{itemize}

Tail-down models
\begin{itemize}
\item allow autocrrelation between both flow-connected and flow-unconnected pairs of sites
\item MA function is only non-zero downstream of a location
\item total hydrologic distance is used for flow-connected pairs, but the hydrologic distances from each site to a common confluence are used for flow-unconnected pairs
\item no need for segment weights since MA function points downstream
\item restriction for relative strength of spatial autocorrelation for each type [flow connected and unconnected]
\begin{itemize}
\item one pair is flow-connected, one pair is not, total distance for flow connected = sum of distances of non-flow connected points to a shared confluence, the strength of the spatial autocorrelation is generally equal or greater for the flow-unconnected pairs
\item none of the current models are able to generate a tail-down (or up) model with significantly stronger spatial autocorrelation between flow-connected pairs than flow-unconnected pairs for an equal hydrologic distance
\item restrictions might make sense for fish populations that have the tendency to invade upstream reaches
\end{itemize}
\end{itemize}

Mixed models
\begin{itemize}
\item variance component approach
\item Covariance mixture = $\sigma^2_{Euc}cor(z_{Euc})+\sigma^2_{TD}cor(z_{TD})+\sigma^2_{TU}cor(z_{TU})+\sigma^2_{nug}I$
\item $cor(z_{TU})= C_{TU} \odot W_{TU}$ elementwise multiplication of flow-connected autocorrelations and the spatial weights matrix 
\item allows for the possibility of more autocorrelation among flow-connected pairs of sites with somewhat less autocorrelation among flow-unconnected pairs of sites
\item multiple range parameters can capture patterns at multiple scales
\end{itemize}

In practice
\begin{itemize}
\item recommend fitting a full covariance matrix, allows the data to determine which variance components have the strongest influence
\item TU model allows discontinuities in predictions at confluences
\item effect of spatial weights on prediction uncertainty is apparent if upstream segments have not been sampled (various combinations can contribute to the downstream observed)
\item possible to produce covariance matrices for Poisson or binomial variables
\end{itemize}

\subsection{}

\end{document}
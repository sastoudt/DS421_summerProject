
\documentclass[12pt]{amsart}
\usepackage{geometry} % see geometry.pdf on how to lay out the page. There's lots.
\geometry{a4paper} % or letter or a5paper or ... etc
% \geometry{landscape} % rotated page geometry
\usepackage{float}
\usepackage{color}
% See the ``Article customise'' template for come common customisations

\title{GAM/WRTDS Comparisons and Observations}
\author{Sara Stoudt}
%\date{} % delete this line to display the current date

%%% BEGIN DOCUMENT
\begin{document}

\maketitle
%\tableofcontents

Note: Filled in D4din, D4nh, D4no23, D6din, D6nh, D6no23, D7din, D7nh, D7no23 by building a mean model on the corresponding dataNiceNoLag data and using fit0.5.

Explanation of the NAs from Marcus: ``The issue is that the flow variable (in this case salinity) used to create the model predictions was slightly different from the flow variable used to fit the model.  It?s actually the same time series in both cases but one was monthly aggregated and the other was not. I had to do some data tweaking early on to make the models comparable.  The NA values occur when the observation date for the flow variable used for prediction did not match the observation date for the aggregated monthly response, i.e., predicting response variables with 'new' data."

\section{Like Table 2 in Beck and Murphy EMA}




% latex table generated in R 3.2.3 by xtable 1.8-2 package
% Sun Aug  7 14:14:01 2016
\begin{table}[H]
\centering
\begin{tabular}{rrrrr}
  \hline
 & GAM\_rmse & WRTDS\_rmse & GAM\_dev & WRTDS\_dev \\ 
  \hline
overall & 4.61 & 6.73 & 21.27 & 45.35 \\ 
  A1 & 1.88 & 2.57 & 3.52 & 6.63 \\ 
  A2 & 1.61 & 2.65 & 2.59 & 7.00 \\ 
  A3 & 2.25 & 3.22 & 5.08 & 10.35 \\ 
  A4 & 2.03 & 2.76 & 4.11 & 7.61 \\ 
  A5 & 2.44 & 3.71 & 5.97 & 13.76 \\ 
  S1 & 2.93 & 3.71 & 8.56 & 13.76 \\ 
  S2 & 2.20 & 3.31 & 4.84 & 10.93 \\ 
  S3 & 1.88 & 3.41 & 3.54 & 11.62 \\ 
  S4 & 2.08 & 3.01 & 4.33 & 9.04 \\ 
  F1 & 2.29 & 3.32 & 5.27 & 11.05 \\ 
  F2 & 1.87 & 2.97 & 0.00 & 0.00 \\ 
  F3 & 2.06 & 3.28 & 0.00 & 0.00 \\ 
  F4 & 2.87 & 3.84 & 8.24 & 14.75 \\ 
   \hline
\end{tabular}
\caption{C10 din} 
\end{table}
% latex table generated in R 3.2.3 by xtable 1.8-2 package
% Sun Aug  7 14:14:01 2016
\begin{table}[H]
\centering
\begin{tabular}{rrrrr}
  \hline
 & GAM\_rmse & WRTDS\_rmse & GAM\_dev & WRTDS\_dev \\ 
  \hline
overall & 11.96 & 16.19 & 143.09 & 262.15 \\ 
  A1 & 4.42 & 5.11 & 19.50 & 26.08 \\ 
  A2 & 4.78 & 5.76 & 22.84 & 33.22 \\ 
  A3 & 6.80 & 8.24 & 46.20 & 67.94 \\ 
  A4 & 5.21 & 6.77 & 27.17 & 45.86 \\ 
  A5 & 5.23 & 9.44 & 27.38 & 89.04 \\ 
  S1 & 6.06 & 7.38 & 36.76 & 54.52 \\ 
  S2 & 6.81 & 8.91 & 46.43 & 79.42 \\ 
  S3 & 5.55 & 9.43 & 30.85 & 88.98 \\ 
  S4 & 5.39 & 6.26 & 29.05 & 39.23 \\ 
  F1 & 5.44 & 8.91 & 29.55 & 79.45 \\ 
  F2 & 6.63 & 8.31 & 0.00 & 0.00 \\ 
  F3 & 5.87 & 7.54 & 0.00 & 0.00 \\ 
  F4 & 5.93 & 7.54 & 35.20 & 56.84 \\ 
   \hline
\end{tabular}
\caption{C10 nh} 
\end{table}
% latex table generated in R 3.2.3 by xtable 1.8-2 package
% Sun Aug  7 14:14:01 2016
\begin{table}[H]
\centering
\begin{tabular}{rrrrr}
  \hline
 & GAM\_rmse & WRTDS\_rmse & GAM\_dev & WRTDS\_dev \\ 
  \hline
overall & 4.60 & 6.95 & 21.17 & 48.31 \\ 
  A1 & 1.97 & 2.70 & 3.90 & 7.32 \\ 
  A2 & 1.57 & 2.74 & 2.46 & 7.52 \\ 
  A3 & 2.17 & 3.28 & 4.72 & 10.76 \\ 
  A4 & 2.06 & 2.91 & 4.24 & 8.44 \\ 
  A5 & 2.42 & 3.78 & 5.84 & 14.27 \\ 
  S1 & 2.89 & 3.90 & 8.36 & 15.23 \\ 
  S2 & 2.24 & 3.43 & 5.01 & 11.75 \\ 
  S3 & 1.88 & 3.46 & 3.55 & 11.94 \\ 
  S4 & 2.06 & 3.06 & 4.25 & 9.39 \\ 
  F1 & 2.30 & 3.35 & 5.28 & 11.23 \\ 
  F2 & 1.93 & 3.08 & 0.00 & 0.00 \\ 
  F3 & 2.03 & 3.39 & 0.00 & 0.00 \\ 
  F4 & 2.84 & 4.01 & 8.05 & 16.12 \\ 
   \hline
\end{tabular}
\caption{C10 no23} 
\end{table}
% latex table generated in R 3.2.3 by xtable 1.8-2 package
% Sun Aug  7 14:14:01 2016
\begin{table}[H]
\centering
\begin{tabular}{rrrrr}
  \hline
 & GAM\_rmse & WRTDS\_rmse & GAM\_dev & WRTDS\_dev \\ 
  \hline
overall & 4.63 & 6.11 & 21.46 & 37.35 \\ 
  A1 & 1.50 & 1.96 & 2.26 & 3.83 \\ 
  A2 & 1.67 & 2.30 & 2.80 & 5.31 \\ 
  A3 & 1.94 & 2.81 & 3.78 & 7.88 \\ 
  A4 & 1.64 & 2.15 & 2.68 & 4.61 \\ 
  A5 & 3.15 & 3.96 & 9.94 & 15.72 \\ 
  S1 & 2.11 & 3.33 & 4.47 & 11.11 \\ 
  S2 & 1.69 & 2.73 & 2.84 & 7.47 \\ 
  S3 & 2.79 & 3.15 & 7.77 & 9.95 \\ 
  S4 & 2.53 & 2.97 & 6.39 & 8.82 \\ 
  F1 & 2.43 & 2.91 & 5.92 & 8.48 \\ 
  F2 & 2.80 & 3.22 & 0.00 & 0.00 \\ 
  F3 & 1.99 & 2.79 & 0.00 & 0.00 \\ 
  F4 & 1.93 & 3.28 & 3.74 & 10.73 \\ 
   \hline
\end{tabular}
\caption{C3 din} 
\end{table}
% latex table generated in R 3.2.3 by xtable 1.8-2 package
% Sun Aug  7 14:14:01 2016
\begin{table}[H]
\centering
\begin{tabular}{rrrrr}
  \hline
 & GAM\_rmse & WRTDS\_rmse & GAM\_dev & WRTDS\_dev \\ 
  \hline
overall & 5.79 & 9.48 & 33.51 & 89.81 \\ 
  A1 & 1.87 & 2.94 & 3.51 & 8.67 \\ 
  A2 & 1.99 & 3.58 & 3.95 & 12.84 \\ 
  A3 & 2.49 & 5.25 & 6.22 & 27.60 \\ 
  A4 & 2.18 & 3.26 & 4.75 & 10.62 \\ 
  A5 & 3.88 & 5.49 & 15.08 & 30.09 \\ 
  S1 & 2.85 & 5.65 & 8.10 & 31.91 \\ 
  S2 & 2.16 & 4.00 & 4.68 & 16.03 \\ 
  S3 & 3.42 & 3.97 & 11.67 & 15.76 \\ 
  S4 & 3.01 & 5.11 & 9.05 & 26.11 \\ 
  F1 & 2.97 & 4.64 & 8.82 & 21.56 \\ 
  F2 & 3.52 & 4.27 & 0.00 & 0.00 \\ 
  F3 & 2.67 & 4.42 & 0.00 & 0.00 \\ 
  F4 & 2.28 & 5.52 & 5.20 & 30.51 \\ 
   \hline
\end{tabular}
\caption{C3 nh} 
\end{table}
% latex table generated in R 3.2.3 by xtable 1.8-2 package
% Sun Aug  7 14:14:01 2016
\begin{table}[H]
\centering
\begin{tabular}{rrrrr}
  \hline
 & GAM\_rmse & WRTDS\_rmse & GAM\_dev & WRTDS\_dev \\ 
  \hline
overall & 5.77 & 7.20 & 33.29 & 51.82 \\ 
  A1 & 1.90 & 2.88 & 3.59 & 8.28 \\ 
  A2 & 2.55 & 2.85 & 6.52 & 8.11 \\ 
  A3 & 2.95 & 3.71 & 8.71 & 13.79 \\ 
  A4 & 1.75 & 2.34 & 3.06 & 5.46 \\ 
  A5 & 3.38 & 4.02 & 11.40 & 16.18 \\ 
  S1 & 2.63 & 3.79 & 6.94 & 14.35 \\ 
  S2 & 2.56 & 3.14 & 6.58 & 9.87 \\ 
  S3 & 3.20 & 3.62 & 10.24 & 13.08 \\ 
  S4 & 3.09 & 3.81 & 9.53 & 14.52 \\ 
  F1 & 3.44 & 3.92 & 11.86 & 15.38 \\ 
  F2 & 3.20 & 3.67 & 0.00 & 0.00 \\ 
  F3 & 2.42 & 3.23 & 0.00 & 0.00 \\ 
  F4 & 2.31 & 3.55 & 5.32 & 12.58 \\ 
   \hline
\end{tabular}
\caption{C3 no23} 
\end{table}
% latex table generated in R 3.2.3 by xtable 1.8-2 package
% Sun Aug  7 14:14:01 2016
\begin{table}[H]
\centering
\begin{tabular}{rrrrr}
  \hline
 & GAM\_rmse & WRTDS\_rmse & GAM\_dev & WRTDS\_dev \\ 
  \hline
overall & 6.44 & 7.69 & 41.49 & 59.11 \\ 
  A1 & 2.60 & 3.22 & 6.75 & 10.39 \\ 
  A2 & 2.66 & 3.55 & 7.09 & 12.62 \\ 
  A3 & 3.16 & 3.31 & 10.00 & 10.97 \\ 
  A4 & 3.08 & 3.40 & 9.47 & 11.57 \\ 
  A5 & 2.86 & 3.68 & 8.18 & 13.57 \\ 
  S1 & 3.03 & 3.50 & 9.21 & 12.23 \\ 
  S2 & 2.98 & 3.40 & 8.91 & 11.58 \\ 
  S3 & 3.15 & 4.29 & 9.90 & 18.41 \\ 
  S4 & 3.67 & 4.11 & 13.46 & 16.89 \\ 
  F1 & 3.28 & 3.76 & 10.76 & 14.17 \\ 
  F2 & 3.46 & 3.78 & 0.00 & 0.00 \\ 
  F3 & 3.36 & 4.29 & 0.00 & 0.00 \\ 
  F4 & 2.73 & 3.49 & 7.43 & 12.21 \\ 
   \hline
\end{tabular}
\caption{P8 din} 
\end{table}
% latex table generated in R 3.2.3 by xtable 1.8-2 package
% Sun Aug  7 14:14:01 2016
\begin{table}[H]
\centering
\begin{tabular}{rrrrr}
  \hline
 & GAM\_rmse & WRTDS\_rmse & GAM\_dev & WRTDS\_dev \\ 
  \hline
overall & 14.19 & 15.64 & 201.42 & 244.50 \\ 
  A1 & 4.23 & 5.04 & 17.88 & 25.43 \\ 
  A2 & 4.93 & 5.44 & 24.35 & 29.63 \\ 
  A3 & 6.88 & 7.13 & 47.33 & 50.88 \\ 
  A4 & 7.92 & 8.27 & 62.68 & 68.44 \\ 
  A5 & 7.01 & 8.37 & 49.17 & 70.12 \\ 
  S1 & 6.86 & 7.24 & 47.11 & 52.37 \\ 
  S2 & 7.51 & 8.35 & 56.33 & 69.68 \\ 
  S3 & 7.15 & 8.23 & 51.18 & 67.80 \\ 
  S4 & 6.84 & 7.39 & 46.80 & 54.64 \\ 
  F1 & 7.32 & 8.06 & 53.64 & 65.01 \\ 
  F2 & 7.55 & 8.18 & 0.00 & 0.00 \\ 
  F3 & 6.78 & 8.30 & 0.00 & 0.00 \\ 
  F4 & 6.69 & 6.61 & 44.78 & 43.69 \\ 
   \hline
\end{tabular}
\caption{P8 nh} 
\end{table}
% latex table generated in R 3.2.3 by xtable 1.8-2 package
% Sun Aug  7 14:14:01 2016
\begin{table}[H]
\centering
\begin{tabular}{rrrrr}
  \hline
 & GAM\_rmse & WRTDS\_rmse & GAM\_dev & WRTDS\_dev \\ 
  \hline
overall & 7.37 & 8.55 & 54.29 & 73.15 \\ 
  A1 & 3.00 & 3.57 & 9.00 & 12.76 \\ 
  A2 & 2.62 & 3.62 & 6.88 & 13.08 \\ 
  A3 & 4.58 & 4.74 & 21.00 & 22.44 \\ 
  A4 & 2.97 & 3.36 & 8.83 & 11.26 \\ 
  A5 & 2.93 & 3.69 & 8.58 & 13.60 \\ 
  S1 & 2.96 & 3.46 & 8.75 & 12.00 \\ 
  S2 & 4.48 & 4.97 & 20.05 & 24.72 \\ 
  S3 & 3.23 & 4.35 & 10.40 & 18.91 \\ 
  S4 & 3.88 & 4.19 & 15.09 & 17.52 \\ 
  F1 & 3.39 & 3.75 & 11.52 & 14.08 \\ 
  F2 & 3.47 & 3.78 & 0.00 & 0.00 \\ 
  F3 & 4.86 & 5.60 & 0.00 & 0.00 \\ 
  F4 & 2.67 & 3.66 & 7.12 & 13.42 \\ 
   \hline
\end{tabular}
\caption{P8 no23} 
\end{table}
% latex table generated in R 3.2.3 by xtable 1.8-2 package
% Sun Aug  7 14:14:01 2016
\begin{table}[H]
\centering
\begin{tabular}{rrrrr}
  \hline
 & GAM\_rmse & WRTDS\_rmse & GAM\_dev & WRTDS\_dev \\ 
  \hline
overall & 5.97 & 7.41 & 35.64 & 54.98 \\ 
  A1 & 3.42 & 4.62 & 11.68 & 21.38 \\ 
  A2 & 3.13 & 3.61 & 9.79 & 13.04 \\ 
  A3 & 2.96 & 3.47 & 8.78 & 12.04 \\ 
  A4 & 0.00 & 0.00 & 0.00 & 0.00 \\ 
  A5 & 2.32 & 2.92 & 5.40 & 8.52 \\ 
  S1 & 2.71 & 3.60 & 7.37 & 12.98 \\ 
  S2 & 3.75 & 4.74 & 14.07 & 22.47 \\ 
  S3 & 3.22 & 3.97 & 10.34 & 15.76 \\ 
  S4 & 1.97 & 1.94 & 3.86 & 3.77 \\ 
  F1 & 2.65 & 3.19 & 7.04 & 10.15 \\ 
  F2 & 3.29 & 4.30 & 0.00 & 0.00 \\ 
  F3 & 2.89 & 3.51 & 0.00 & 0.00 \\ 
  F4 & 3.07 & 3.75 & 9.43 & 14.04 \\ 
   \hline
\end{tabular}
\caption{D19 din} 
\end{table}
% latex table generated in R 3.2.3 by xtable 1.8-2 package
% Sun Aug  7 14:14:01 2016
\begin{table}[H]
\centering
\begin{tabular}{rrrrr}
  \hline
 & GAM\_rmse & WRTDS\_rmse & GAM\_dev & WRTDS\_dev \\ 
  \hline
overall & 7.33 & 8.15 & 53.67 & 66.48 \\ 
  A1 & 3.68 & 4.06 & 13.52 & 16.50 \\ 
  A2 & 3.03 & 3.38 & 9.16 & 11.44 \\ 
  A3 & 3.68 & 3.94 & 13.55 & 15.51 \\ 
  A4 & 0.00 & 0.00 & 0.00 & 0.00 \\ 
  A5 & 4.18 & 4.80 & 17.44 & 23.02 \\ 
  S1 & 3.13 & 3.48 & 9.83 & 12.10 \\ 
  S2 & 4.60 & 5.22 & 21.15 & 27.27 \\ 
  S3 & 3.78 & 4.07 & 14.26 & 16.60 \\ 
  S4 & 2.90 & 3.24 & 8.44 & 10.51 \\ 
  F1 & 3.98 & 4.56 & 15.83 & 20.77 \\ 
  F2 & 4.16 & 4.41 & 0.00 & 0.00 \\ 
  F3 & 3.17 & 3.62 & 0.00 & 0.00 \\ 
  F4 & 3.24 & 3.63 & 10.47 & 13.20 \\ 
   \hline
\end{tabular}
\caption{D19 nh} 
\end{table}
% latex table generated in R 3.2.3 by xtable 1.8-2 package
% Sun Aug  7 14:14:01 2016
\begin{table}[H]
\centering
\begin{tabular}{rrrrr}
  \hline
 & GAM\_rmse & WRTDS\_rmse & GAM\_dev & WRTDS\_dev \\ 
  \hline
overall & 7.25 & 8.65 & 52.52 & 74.90 \\ 
  A1 & 4.63 & 5.84 & 21.44 & 34.08 \\ 
  A2 & 3.98 & 4.34 & 15.82 & 18.80 \\ 
  A3 & 3.16 & 3.62 & 10.00 & 13.07 \\ 
  A4 & 0.00 & 0.00 & 0.00 & 0.00 \\ 
  A5 & 2.29 & 2.99 & 5.26 & 8.95 \\ 
  S1 & 3.07 & 3.97 & 9.40 & 15.74 \\ 
  S2 & 4.81 & 5.78 & 23.17 & 33.38 \\ 
  S3 & 3.93 & 4.59 & 15.45 & 21.08 \\ 
  S4 & 2.12 & 2.17 & 4.49 & 4.70 \\ 
  F1 & 2.81 & 3.36 & 7.90 & 11.29 \\ 
  F2 & 4.45 & 5.47 & 0.00 & 0.00 \\ 
  F3 & 3.36 & 3.90 & 0.00 & 0.00 \\ 
  F4 & 3.68 & 4.30 & 13.54 & 18.49 \\ 
   \hline
\end{tabular}
\caption{D19 no23} 
\end{table}
% latex table generated in R 3.2.3 by xtable 1.8-2 package
% Sun Aug  7 14:14:01 2016
\begin{table}[H]
\centering
\begin{tabular}{rrrrr}
  \hline
 & GAM\_rmse & WRTDS\_rmse & GAM\_dev & WRTDS\_dev \\ 
  \hline
overall & 4.54 & 5.67 & 20.60 & 32.12 \\ 
  A1 & 2.06 & 2.97 & 4.23 & 8.83 \\ 
  A2 & 2.24 & 2.63 & 5.04 & 6.89 \\ 
  A3 & 2.21 & 2.69 & 4.90 & 7.21 \\ 
  A4 & 1.81 & 1.89 & 3.27 & 3.58 \\ 
  A5 & 1.78 & 2.37 & 3.17 & 5.61 \\ 
  S1 & 2.53 & 3.29 & 6.38 & 10.83 \\ 
  S2 & 2.64 & 3.19 & 6.96 & 10.21 \\ 
  S3 & 2.13 & 2.72 & 4.53 & 7.39 \\ 
  S4 & 1.65 & 1.92 & 2.73 & 3.70 \\ 
  F1 & 2.34 & 2.97 & 5.48 & 8.83 \\ 
  F2 & 2.03 & 2.58 & 0.00 & 0.00 \\ 
  F3 & 2.19 & 2.64 & 0.00 & 0.00 \\ 
  F4 & 2.49 & 3.11 & 6.22 & 9.68 \\ 
   \hline
\end{tabular}
\caption{D26 din} 
\end{table}
% latex table generated in R 3.2.3 by xtable 1.8-2 package
% Sun Aug  7 14:14:01 2016
\begin{table}[H]
\centering
\begin{tabular}{rrrrr}
  \hline
 & GAM\_rmse & WRTDS\_rmse & GAM\_dev & WRTDS\_dev \\ 
  \hline
overall & 6.09 & 6.92 & 37.05 & 47.87 \\ 
  A1 & 2.32 & 2.64 & 5.39 & 6.95 \\ 
  A2 & 2.69 & 2.96 & 7.22 & 8.79 \\ 
  A3 & 3.48 & 4.14 & 12.11 & 17.16 \\ 
  A4 & 2.42 & 2.69 & 5.84 & 7.23 \\ 
  A5 & 2.55 & 2.78 & 6.50 & 7.75 \\ 
  S1 & 2.41 & 2.57 & 5.80 & 6.61 \\ 
  S2 & 4.00 & 4.64 & 15.97 & 21.57 \\ 
  S3 & 3.09 & 3.67 & 9.52 & 13.50 \\ 
  S4 & 2.40 & 2.49 & 5.76 & 6.20 \\ 
  F1 & 3.52 & 4.14 & 12.38 & 17.16 \\ 
  F2 & 2.56 & 3.09 & 0.00 & 0.00 \\ 
  F3 & 3.19 & 3.57 & 0.00 & 0.00 \\ 
  F4 & 2.82 & 2.90 & 7.96 & 8.44 \\ 
   \hline
\end{tabular}
\caption{D26 nh} 
\end{table}
% latex table generated in R 3.2.3 by xtable 1.8-2 package
% Sun Aug  7 14:14:01 2016
\begin{table}[H]
\centering
\begin{tabular}{rrrrr}
  \hline
 & GAM\_rmse & WRTDS\_rmse & GAM\_dev & WRTDS\_dev \\ 
  \hline
overall & 4.92 & 6.21 & 24.25 & 38.51 \\ 
  A1 & 2.23 & 3.28 & 4.99 & 10.75 \\ 
  A2 & 2.41 & 2.80 & 5.82 & 7.85 \\ 
  A3 & 2.37 & 2.91 & 5.60 & 8.46 \\ 
  A4 & 1.94 & 2.10 & 3.76 & 4.41 \\ 
  A5 & 2.02 & 2.65 & 4.07 & 7.04 \\ 
  S1 & 2.84 & 3.78 & 8.08 & 14.28 \\ 
  S2 & 2.79 & 3.34 & 7.77 & 11.17 \\ 
  S3 & 2.23 & 2.85 & 4.97 & 8.12 \\ 
  S4 & 1.85 & 2.23 & 3.43 & 4.95 \\ 
  F1 & 2.44 & 3.05 & 5.94 & 9.28 \\ 
  F2 & 2.20 & 2.91 & 0.00 & 0.00 \\ 
  F3 & 2.43 & 2.97 & 0.00 & 0.00 \\ 
  F4 & 2.75 & 3.46 & 7.57 & 11.96 \\ 
   \hline
\end{tabular}
\caption{D26 no23} 
\end{table}
% latex table generated in R 3.2.3 by xtable 1.8-2 package
% Sun Aug  7 14:14:01 2016
\begin{table}[H]
\centering
\begin{tabular}{rrrrr}
  \hline
 & GAM\_rmse & WRTDS\_rmse & GAM\_dev & WRTDS\_dev \\ 
  \hline
overall & 7.80 & 9.43 & 60.82 & 88.87 \\ 
  A1 & 3.97 & 5.09 & 15.78 & 25.87 \\ 
  A2 & 3.31 & 3.85 & 10.96 & 14.83 \\ 
  A3 & 4.21 & 4.63 & 17.68 & 21.41 \\ 
  A4 & 2.82 & 3.49 & 7.93 & 12.15 \\ 
  A5 & 2.91 & 3.82 & 8.47 & 14.61 \\ 
  S1 & 3.43 & 4.57 & 11.78 & 20.92 \\ 
  S2 & 4.53 & 5.39 & 20.50 & 29.02 \\ 
  S3 & 4.40 & 5.23 & 19.39 & 27.38 \\ 
  S4 & 3.02 & 3.40 & 9.14 & 11.54 \\ 
  F1 & 3.90 & 4.53 & 15.23 & 20.54 \\ 
  F2 & 4.16 & 5.24 & 0.00 & 0.00 \\ 
  F3 & 3.83 & 4.49 & 0.00 & 0.00 \\ 
  F4 & 3.70 & 4.55 & 13.68 & 20.74 \\ 
   \hline
\end{tabular}
\caption{D28 din} 
\end{table}
% latex table generated in R 3.2.3 by xtable 1.8-2 package
% Sun Aug  7 14:14:01 2016
\begin{table}[H]
\centering
\begin{tabular}{rrrrr}
  \hline
 & GAM\_rmse & WRTDS\_rmse & GAM\_dev & WRTDS\_dev \\ 
  \hline
overall & 8.57 & 10.71 & 73.49 & 114.63 \\ 
  A1 & 4.11 & 5.17 & 16.90 & 26.69 \\ 
  A2 & 3.83 & 4.91 & 14.65 & 24.15 \\ 
  A3 & 3.53 & 4.51 & 12.46 & 20.30 \\ 
  A4 & 3.80 & 4.52 & 14.42 & 20.46 \\ 
  A5 & 3.88 & 4.80 & 15.06 & 23.02 \\ 
  S1 & 4.09 & 5.69 & 16.71 & 32.38 \\ 
  S2 & 4.58 & 5.98 & 21.00 & 35.70 \\ 
  S3 & 4.12 & 4.76 & 16.96 & 22.65 \\ 
  S4 & 4.34 & 4.89 & 18.81 & 23.88 \\ 
  F1 & 4.45 & 5.66 & 19.81 & 32.03 \\ 
  F2 & 4.30 & 5.22 & 0.00 & 0.00 \\ 
  F3 & 3.90 & 4.37 & 0.00 & 0.00 \\ 
  F4 & 4.47 & 6.02 & 20.00 & 36.21 \\ 
   \hline
\end{tabular}
\caption{D28 nh} 
\end{table}
% latex table generated in R 3.2.3 by xtable 1.8-2 package
% Sun Aug  7 14:14:01 2016
\begin{table}[H]
\centering
\begin{tabular}{rrrrr}
  \hline
 & GAM\_rmse & WRTDS\_rmse & GAM\_dev & WRTDS\_dev \\ 
  \hline
overall & 9.06 & 10.54 & 82.00 & 111.19 \\ 
  A1 & 4.78 & 5.88 & 22.81 & 34.62 \\ 
  A2 & 4.25 & 4.66 & 18.03 & 21.69 \\ 
  A3 & 4.67 & 5.03 & 21.84 & 25.28 \\ 
  A4 & 2.94 & 3.53 & 8.67 & 12.45 \\ 
  A5 & 3.26 & 4.14 & 10.66 & 17.15 \\ 
  S1 & 3.43 & 4.24 & 11.74 & 18.02 \\ 
  S2 & 5.84 & 6.79 & 34.08 & 46.08 \\ 
  S3 & 5.12 & 5.88 & 26.23 & 34.62 \\ 
  S4 & 3.15 & 3.53 & 9.95 & 12.48 \\ 
  F1 & 4.45 & 5.07 & 19.84 & 25.70 \\ 
  F2 & 5.04 & 6.05 & 0.00 & 0.00 \\ 
  F3 & 4.31 & 4.90 & 0.00 & 0.00 \\ 
  F4 & 4.26 & 4.98 & 18.18 & 24.80 \\ 
   \hline
\end{tabular}
\caption{D28 no23} 
\end{table}



\section{Like Tables 3 \& 4 in Beck and Murphy EMA}


% latex table generated in R 3.2.3 by xtable 1.8-2 package
% Sun Aug  7 14:13:27 2016
\begin{table}[H]
\centering
\begin{tabular}{rrrrr}
  \hline
 & GAM\_avg & WRTDS\_avg & GAM\_pc & WRTDS\_pc \\ 
  \hline
overall & 0.21 & 0.13 & 1.07 & -3.83 \\ 
  annual1 & 0.10 & 0.06 & -4.77 & 2.95 \\ 
  annual2 & 0.23 & 0.06 & -4.41 & 3.05 \\ 
  annual3 & 0.28 & 0.28 & -7.51 & 7.15 \\ 
  annual4 & 0.37 & 0.28 & -4.77 & 2.95 \\ 
  annual5 & 0.08 & 0.02 & -4.77 & 2.95 \\ 
  seasonal1 & 0.30 & 0.30 & 4.40 & -5.09 \\ 
  seasonal2 & -0.15 & -0.15 & 4.05 & -3.09 \\ 
  seasonal3 & 0.29 & 0.29 & 2.64 & -3.31 \\ 
  seasonal4 & 0.39 & 0.39 & 2.67 & -3.22 \\ 
  flow1 & 0.44 & 0.43 & 2.64 & -3.31 \\ 
  flow2 & 0.59 & 0.44 & 3.36 & -3.49 \\ 
  flow3 & 0.35 & 0.21 & 2.67 & -3.22 \\ 
  flow4 & -0.53 & -0.53 & 2.64 & -3.31 \\ 
   \hline
\end{tabular}
\caption{C10 din} 
\end{table}
% latex table generated in R 3.2.3 by xtable 1.8-2 package
% Sun Aug  7 14:13:27 2016
\begin{table}[H]
\centering
\begin{tabular}{rrrrr}
  \hline
 & GAM\_avg & WRTDS\_avg & GAM\_pc & WRTDS\_pc \\ 
  \hline
overall & -3.10 & -3.36 & 0.84 & 1.45 \\ 
  annual1 & -2.22 & -2.24 & 0.50 & 0.45 \\ 
  annual2 & -2.59 & -2.77 & 0.52 & 0.58 \\ 
  annual3 & -3.01 & -3.23 & 0.97 & 0.95 \\ 
  annual4 & -3.47 & -3.75 & 0.50 & 0.45 \\ 
  annual5 & -3.94 & -4.42 & 0.50 & 0.45 \\ 
  seasonal1 & -2.34 & -2.34 & 0.02 & 0.36 \\ 
  seasonal2 & -3.65 & -3.65 & 0.50 & 0.81 \\ 
  seasonal3 & -3.69 & -3.69 & 0.67 & 1.15 \\ 
  seasonal4 & -2.71 & -2.71 & 1.04 & 1.36 \\ 
  flow1 & -3.37 & -3.67 & 0.67 & 1.15 \\ 
  flow2 & -3.05 & -3.35 & -0.40 & 0.18 \\ 
  flow3 & -2.91 & -3.20 & 1.04 & 1.36 \\ 
  flow4 & -3.07 & -3.22 & 0.67 & 1.15 \\ 
   \hline
\end{tabular}
\caption{C10 nh} 
\end{table}
% latex table generated in R 3.2.3 by xtable 1.8-2 package
% Sun Aug  7 14:13:27 2016
\begin{table}[H]
\centering
\begin{tabular}{rrrrr}
  \hline
 & GAM\_avg & WRTDS\_avg & GAM\_pc & WRTDS\_pc \\ 
  \hline
overall & 0.14 & 0.06 & -6.54 & -1.86 \\ 
  annual1 & -0.04 & -0.08 & 18.95 & 0.88 \\ 
  annual2 & 0.15 & -0.03 & 18.95 & 0.88 \\ 
  annual3 & 0.21 & 0.22 & 27.56 & 1.53 \\ 
  annual4 & 0.34 & 0.24 & 18.95 & 0.88 \\ 
  annual5 & 0.06 & -0.01 & 18.95 & 0.88 \\ 
  seasonal1 & 0.19 & 0.19 & -17.14 & -2.05 \\ 
  seasonal2 & -0.20 & -0.20 & -17.32 & -1.49 \\ 
  seasonal3 & 0.27 & 0.27 & -12.66 & -1.68 \\ 
  seasonal4 & 0.30 & 0.30 & -12.66 & -1.68 \\ 
  flow1 & 0.39 & 0.38 & -17.32 & -1.49 \\ 
  flow2 & 0.54 & 0.38 & -17.32 & -1.49 \\ 
  flow3 & 0.29 & 0.14 & -17.32 & -1.49 \\ 
  flow4 & -0.65 & -0.64 & -12.66 & -1.68 \\ 
   \hline
\end{tabular}
\caption{C10 no23} 
\end{table}
% latex table generated in R 3.2.3 by xtable 1.8-2 package
% Sun Aug  7 14:13:27 2016
\begin{table}[H]
\centering
\begin{tabular}{rrrrr}
  \hline
 & GAM\_avg & WRTDS\_avg & GAM\_pc & WRTDS\_pc \\ 
  \hline
overall & -1.00 & -1.03 & -0.71 & -0.68 \\ 
  annual1 & -1.21 & -1.26 & 0.27 & 0.48 \\ 
  annual2 & -1.14 & -1.13 & 0.42 & 0.63 \\ 
  annual3 & -0.96 & -0.94 & 0.13 & 0.34 \\ 
  annual4 & -0.91 & -0.97 & 0.27 & 0.48 \\ 
  annual5 & -0.84 & -0.89 & 0.27 & 0.48 \\ 
  seasonal1 & -0.98 & -0.98 & -0.12 & 0.01 \\ 
  seasonal2 & -1.08 & -1.08 & 0.13 & 0.30 \\ 
  seasonal3 & -1.18 & -1.18 & 0.13 & 0.30 \\ 
  seasonal4 & -0.75 & -0.75 & 0.19 & 0.28 \\ 
  flow1 & -0.71 & -0.75 & 0.19 & 0.28 \\ 
  flow2 & -0.88 & -0.93 & -0.12 & 0.01 \\ 
  flow3 & -1.06 & -1.10 & 0.13 & 0.30 \\ 
  flow4 & -1.34 & -1.33 & 0.13 & 0.30 \\ 
   \hline
\end{tabular}
\caption{C3 din} 
\end{table}
% latex table generated in R 3.2.3 by xtable 1.8-2 package
% Sun Aug  7 14:13:27 2016
\begin{table}[H]
\centering
\begin{tabular}{rrrrr}
  \hline
 & GAM\_avg & WRTDS\_avg & GAM\_pc & WRTDS\_pc \\ 
  \hline
overall & -1.61 & -1.65 & -0.69 & -0.66 \\ 
  annual1 & -1.96 & -1.99 & 0.39 & 0.42 \\ 
  annual2 & -1.74 & -1.78 & 0.43 & 0.53 \\ 
  annual3 & -1.59 & -1.52 & 0.39 & 0.35 \\ 
  annual4 & -1.43 & -1.57 & 0.39 & 0.42 \\ 
  annual5 & -1.41 & -1.48 & 0.39 & 0.42 \\ 
  seasonal1 & -1.93 & -1.93 & -0.26 & -0.18 \\ 
  seasonal2 & -1.64 & -1.64 & -0.19 & -0.17 \\ 
  seasonal3 & -1.57 & -1.57 & -0.19 & -0.17 \\ 
  seasonal4 & -1.32 & -1.32 & -0.10 & -0.17 \\ 
  flow1 & -1.11 & -1.16 & -0.10 & -0.17 \\ 
  flow2 & -1.29 & -1.44 & -0.26 & -0.18 \\ 
  flow3 & -1.60 & -1.67 & -0.19 & -0.17 \\ 
  flow4 & -2.42 & -2.33 & -0.19 & -0.17 \\ 
   \hline
\end{tabular}
\caption{C3 nh} 
\end{table}
% latex table generated in R 3.2.3 by xtable 1.8-2 package
% Sun Aug  7 14:13:27 2016
\begin{table}[H]
\centering
\begin{tabular}{rrrrr}
  \hline
 & GAM\_avg & WRTDS\_avg & GAM\_pc & WRTDS\_pc \\ 
  \hline
overall & -1.96 & -1.99 & -0.21 & -0.03 \\ 
  annual1 & -1.97 & -2.04 & -0.11 & 0.13 \\ 
  annual2 & -2.11 & -2.07 & -0.19 & 0.04 \\ 
  annual3 & -1.97 & -2.03 & -0.11 & 0.13 \\ 
  annual4 & -2.01 & -1.98 & -0.11 & 0.13 \\ 
  annual5 & -1.80 & -1.86 & -0.11 & 0.13 \\ 
  seasonal1 & -1.59 & -1.59 & 0.14 & 0.32 \\ 
  seasonal2 & -2.07 & -2.07 & 0.36 & 0.59 \\ 
  seasonal3 & -2.39 & -2.39 & 0.36 & 0.59 \\ 
  seasonal4 & -1.78 & -1.78 & 0.22 & 0.48 \\ 
  flow1 & -1.95 & -1.96 & 0.22 & 0.48 \\ 
  flow2 & -2.00 & -1.99 & 0.11 & 0.17 \\ 
  flow3 & -2.07 & -2.09 & 0.36 & 0.59 \\ 
  flow4 & -1.82 & -1.90 & 0.38 & 0.63 \\ 
   \hline
\end{tabular}
\caption{C3 no23} 
\end{table}
% latex table generated in R 3.2.3 by xtable 1.8-2 package
% Sun Aug  7 14:13:27 2016
\begin{table}[H]
\centering
\begin{tabular}{rrrrr}
  \hline
 & GAM\_avg & WRTDS\_avg & GAM\_pc & WRTDS\_pc \\ 
  \hline
overall & 0.31 & 0.23 & 1.50 & -37.20 \\ 
  annual1 & -0.04 & -0.08 & -2.09 & 11.63 \\ 
  annual2 & 0.23 & 0.08 & -3.02 & 34.90 \\ 
  annual3 & 0.33 & 0.33 & -2.69 & 23.98 \\ 
  annual4 & 0.52 & 0.44 & -3.51 & 31.67 \\ 
  annual5 & 0.42 & 0.34 & -2.09 & 11.63 \\ 
  seasonal1 & 0.53 & 0.53 & -0.25 & 1.01 \\ 
  seasonal2 & 0.04 & 0.04 & -0.25 & 1.01 \\ 
  seasonal3 & 0.07 & 0.07 & -0.12 & 0.39 \\ 
  seasonal4 & 0.58 & 0.58 & -0.25 & 1.01 \\ 
  flow1 & 0.33 & 0.34 & -0.42 & 4.53 \\ 
  flow2 & 0.66 & 0.54 & 0.55 & -14.27 \\ 
  flow3 & 0.56 & 0.39 & -0.12 & 0.39 \\ 
  flow4 & -0.32 &  & -0.42 & 4.53 \\ 
   \hline
\end{tabular}
\caption{P8 din} 
\end{table}
% latex table generated in R 3.2.3 by xtable 1.8-2 package
% Sun Aug  7 14:13:27 2016
\begin{table}[H]
\centering
\begin{tabular}{rrrrr}
  \hline
 & GAM\_avg & WRTDS\_avg & GAM\_pc & WRTDS\_pc \\ 
  \hline
overall & -1.89 & -2.00 & 0.87 & 0.50 \\ 
  annual1 & -2.13 & -2.24 & 0.54 & 0.07 \\ 
  annual2 & -1.73 & -1.84 & 0.99 & 0.38 \\ 
  annual3 & -1.67 & -1.70 & 0.85 & 0.26 \\ 
  annual4 & -1.48 & -1.65 & 0.95 & 0.35 \\ 
  annual5 & -2.26 & -2.40 & 0.54 & 0.07 \\ 
  seasonal1 & -1.25 & -1.25 & 0.79 & 0.60 \\ 
  seasonal2 & -2.33 & -2.33 & 0.79 & 0.60 \\ 
  seasonal3 & -2.59 & -2.59 & 0.74 & 0.35 \\ 
  seasonal4 & -1.37 & -1.37 & 0.79 & 0.60 \\ 
  flow1 & -2.32 & -2.28 & 0.83 & 0.57 \\ 
  flow2 & -1.56 & -1.73 & 0.27 & 0.10 \\ 
  flow3 & -1.40 & -1.69 & 0.74 & 0.35 \\ 
  flow4 & -2.25 &  & 0.83 & 0.57 \\ 
   \hline
\end{tabular}
\caption{P8 nh} 
\end{table}
% latex table generated in R 3.2.3 by xtable 1.8-2 package
% Sun Aug  7 14:13:27 2016
\begin{table}[H]
\centering
\begin{tabular}{rrrrr}
  \hline
 & GAM\_avg & WRTDS\_avg & GAM\_pc & WRTDS\_pc \\ 
  \hline
overall & 0.12 & 0.05 & 2.92 & -4.48 \\ 
  annual1 & -0.22 & -0.25 & -3.96 & 1.17 \\ 
  annual2 & 0.03 & -0.13 & -4.96 & 2.38 \\ 
  annual3 & 0.09 & 0.10 & -3.96 & 1.17 \\ 
  annual4 & 0.30 & 0.23 & -5.84 & 3.46 \\ 
  annual5 & 0.30 & 0.23 & -3.96 & 1.17 \\ 
  seasonal1 & 0.25 & 0.25 & -0.16 & -0.55 \\ 
  seasonal2 & -0.11 & -0.11 & 0.33 & -1.18 \\ 
  seasonal3 & -0.03 & -0.03 & -0.37 & -0.13 \\ 
  seasonal4 & 0.36 & 0.36 & 0.33 & -1.18 \\ 
  flow1 & 0.22 & 0.22 & -0.16 & -0.55 \\ 
  flow2 & 0.45 & 0.36 & 1.73 & -1.73 \\ 
  flow3 & 0.33 & 0.18 & -0.38 & -0.40 \\ 
  flow4 & -0.52 &  & -0.16 & -0.55 \\ 
   \hline
\end{tabular}
\caption{P8 no23} 
\end{table}
% latex table generated in R 3.2.3 by xtable 1.8-2 package
% Sun Aug  7 14:13:27 2016
\begin{table}[H]
\centering
\begin{tabular}{rrrrr}
  \hline
 & GAM\_avg & WRTDS\_avg & GAM\_pc & WRTDS\_pc \\ 
  \hline
overall & -1.02 & -1.01 & -0.62 & 0.84 \\ 
  annual1 & -1.04 & -1.04 & -0.73 & 0.21 \\ 
  annual2 & -1.14 & -1.04 & -0.65 & 0.21 \\ 
  annual3 & -0.89 & -0.93 & 0.07 & 4.02 \\ 
  annual4 &  &  &  &  \\ 
  annual5 & -1.00 & -1.03 & -0.73 & 0.21 \\ 
  seasonal1 & -0.46 & -0.46 & 0.41 & 4.21 \\ 
  seasonal2 & -1.19 & -1.19 & -0.25 & 2.14 \\ 
  seasonal3 & -1.57 & -1.57 & 0.34 & 3.96 \\ 
  seasonal4 & -0.85 & -0.85 & 0.34 & 3.96 \\ 
  flow1 & -0.87 & -0.92 & 0.41 & 4.21 \\ 
  flow2 & -1.13 & -1.07 & 0.02 & 3.26 \\ 
  flow3 & -1.26 & -1.23 & 0.34 & 3.96 \\ 
  flow4 & -0.81 & -0.84 & 0.02 & 3.26 \\ 
   \hline
\end{tabular}
\caption{D19 din} 
\end{table}
% latex table generated in R 3.2.3 by xtable 1.8-2 package
% Sun Aug  7 14:13:27 2016
\begin{table}[H]
\centering
\begin{tabular}{rrrrr}
  \hline
 & GAM\_avg & WRTDS\_avg & GAM\_pc & WRTDS\_pc \\ 
  \hline
overall & -3.12 & -3.14 & -0.03 & 0.14 \\ 
  annual1 & -3.12 & -3.13 & -0.08 & -0.03 \\ 
  annual2 & -3.09 & -3.05 & 0.00 & -0.01 \\ 
  annual3 & -3.02 & -3.06 & 0.55 & 0.65 \\ 
  annual4 &  &  &  &  \\ 
  annual5 & -3.22 & -3.26 & -0.08 & -0.03 \\ 
  seasonal1 & -2.35 & -2.35 & 0.60 & 0.64 \\ 
  seasonal2 & -3.48 & -3.48 & 0.22 & 0.25 \\ 
  seasonal3 & -3.84 & -3.84 & 0.56 & 0.57 \\ 
  seasonal4 & -2.81 & -2.81 & 0.56 & 0.57 \\ 
  flow1 & -3.08 & -3.12 & 0.60 & 0.64 \\ 
  flow2 & -3.26 & -3.23 & 0.47 & 0.44 \\ 
  flow3 & -3.38 & -3.40 & 0.56 & 0.57 \\ 
  flow4 & -2.78 & -2.81 & 0.47 & 0.44 \\ 
   \hline
\end{tabular}
\caption{D19 nh} 
\end{table}
% latex table generated in R 3.2.3 by xtable 1.8-2 package
% Sun Aug  7 14:13:27 2016
\begin{table}[H]
\centering
\begin{tabular}{rrrrr}
  \hline
 & GAM\_avg & WRTDS\_avg & GAM\_pc & WRTDS\_pc \\ 
  \hline
overall & -1.19 & -1.18 & -0.47 & 0.58 \\ 
  annual1 & -1.21 & -1.21 & -0.58 & 0.23 \\ 
  annual2 & -1.38 & -1.25 & -0.58 & 0.23 \\ 
  annual3 & -1.03 & -1.08 & -0.57 & -0.39 \\ 
  annual4 &  &  &  &  \\ 
  annual5 & -1.13 & -1.16 & -0.58 & 0.23 \\ 
  seasonal1 & -0.65 & -0.65 & -0.48 & 0.17 \\ 
  seasonal2 & -1.35 & -1.35 & 0.38 & 2.31 \\ 
  seasonal3 & -1.70 & -1.70 & 0.48 & 2.71 \\ 
  seasonal4 & -1.03 & -1.03 & 0.38 & 2.31 \\ 
  flow1 & -1.03 & -1.08 & 0.24 & 1.68 \\ 
  flow2 & -1.31 & -1.24 & 0.24 & 1.68 \\ 
  flow3 & -1.41 & -1.37 & 0.38 & 2.31 \\ 
  flow4 & -1.00 & -1.02 & 0.38 & 2.31 \\ 
   \hline
\end{tabular}
\caption{D19 no23} 
\end{table}
% latex table generated in R 3.2.3 by xtable 1.8-2 package
% Sun Aug  7 14:13:27 2016
\begin{table}[H]
\centering
\begin{tabular}{rrrrr}
  \hline
 & GAM\_avg & WRTDS\_avg & GAM\_pc & WRTDS\_pc \\ 
  \hline
overall & -0.92 & -0.92 & -0.54 & 0.22 \\ 
  annual1 & -0.96 & -0.96 & -0.37 & 0.44 \\ 
  annual2 & -1.08 & -0.99 & -0.16 & 0.51 \\ 
  annual3 & -0.83 & -0.85 & 0.04 & 1.07 \\ 
  annual4 & -0.86 & -0.88 & -0.51 & 0.19 \\ 
  annual5 & -0.89 & -0.93 & -0.37 & 0.44 \\ 
  seasonal1 & -0.56 & -0.56 & 0.74 & 2.27 \\ 
  seasonal2 & -1.02 & -1.02 & 0.36 & 1.51 \\ 
  seasonal3 & -1.35 & -1.35 & 0.64 & 1.96 \\ 
  seasonal4 & -0.75 & -0.75 & 0.64 & 1.96 \\ 
  flow1 & -0.78 & -0.81 & 0.74 & 2.27 \\ 
  flow2 & -1.00 & -0.96 & 0.36 & 1.51 \\ 
  flow3 & -1.08 & -1.08 & 0.74 & 2.27 \\ 
  flow4 & -0.83 & -0.83 & 0.36 & 1.51 \\ 
   \hline
\end{tabular}
\caption{D26 din} 
\end{table}
% latex table generated in R 3.2.3 by xtable 1.8-2 package
% Sun Aug  7 14:13:27 2016
\begin{table}[H]
\centering
\begin{tabular}{rrrrr}
  \hline
 & GAM\_avg & WRTDS\_avg & GAM\_pc & WRTDS\_pc \\ 
  \hline
overall & -2.40 & -2.41 & -0.36 & -0.08 \\ 
  annual1 & -2.57 & -2.58 & -0.06 & 0.08 \\ 
  annual2 & -2.47 & -2.42 & -0.03 & 0.11 \\ 
  annual3 & -2.32 & -2.30 & 0.03 & 0.19 \\ 
  annual4 & -2.27 & -2.34 & -0.13 & 0.04 \\ 
  annual5 & -2.40 & -2.42 & -0.06 & 0.08 \\ 
  seasonal1 & -2.12 & -2.12 & 0.23 & 0.35 \\ 
  seasonal2 & -2.72 & -2.72 & 0.07 & 0.13 \\ 
  seasonal3 & -2.68 & -2.68 & 0.13 & 0.24 \\ 
  seasonal4 & -2.10 & -2.10 & 0.13 & 0.24 \\ 
  flow1 & -2.28 & -2.30 & 0.23 & 0.35 \\ 
  flow2 & -2.39 & -2.43 & 0.07 & 0.13 \\ 
  flow3 & -2.52 & -2.52 & 0.23 & 0.35 \\ 
  flow4 & -2.42 & -2.39 & 0.07 & 0.13 \\ 
   \hline
\end{tabular}
\caption{D26 nh} 
\end{table}
% latex table generated in R 3.2.3 by xtable 1.8-2 package
% Sun Aug  7 14:13:27 2016
\begin{table}[H]
\centering
\begin{tabular}{rrrrr}
  \hline
 & GAM\_avg & WRTDS\_avg & GAM\_pc & WRTDS\_pc \\ 
  \hline
overall & -1.20 & -1.20 & -0.36 & 0.28 \\ 
  annual1 & -1.21 & -1.20 & -0.34 & 0.32 \\ 
  annual2 & -1.39 & -1.29 & -0.39 & 0.15 \\ 
  annual3 & -1.13 & -1.15 & -0.39 & 0.15 \\ 
  annual4 & -1.16 & -1.16 & -0.18 & 0.88 \\ 
  annual5 & -1.15 & -1.20 & -0.34 & 0.32 \\ 
  seasonal1 & -0.82 & -0.82 & 0.34 & 0.93 \\ 
  seasonal2 & -1.24 & -1.24 & 0.57 & 1.60 \\ 
  seasonal3 & -1.68 & -1.68 & 0.57 & 1.60 \\ 
  seasonal4 & -1.07 & -1.07 & 0.38 & 1.40 \\ 
  flow1 & -1.06 & -1.10 & 0.34 & 0.93 \\ 
  flow2 & -1.31 & -1.25 & 0.38 & 1.40 \\ 
  flow3 & -1.38 & -1.37 & 0.57 & 1.60 \\ 
  flow4 & -1.06 & -1.08 & 0.57 & 1.60 \\ 
   \hline
\end{tabular}
\caption{D26 no23} 
\end{table}
% latex table generated in R 3.2.3 by xtable 1.8-2 package
% Sun Aug  7 14:13:27 2016
\begin{table}[H]
\centering
\begin{tabular}{rrrrr}
  \hline
 & GAM\_avg & WRTDS\_avg & GAM\_pc & WRTDS\_pc \\ 
  \hline
overall & -1.03 & -1.04 & -0.52 & 2.49 \\ 
  annual1 & -0.97 & -0.97 & -0.86 & 0.44 \\ 
  annual2 & -1.14 & -1.06 & 0.23 & 5.13 \\ 
  annual3 & -0.92 & -0.95 & 0.23 & 5.13 \\ 
  annual4 & -0.90 & -0.94 & -0.09 & 3.33 \\ 
  annual5 & -1.17 & -1.20 & -0.86 & 0.44 \\ 
  seasonal1 & -0.39 & -0.39 & -0.62 & 1.21 \\ 
  seasonal2 & -1.05 & -1.05 & -0.38 & 0.53 \\ 
  seasonal3 & -1.80 & -1.80 & -0.08 & 2.65 \\ 
  seasonal4 & -0.91 & -0.91 & -0.70 & 0.40 \\ 
  flow1 & -0.99 & -1.04 & -0.38 & 0.53 \\ 
  flow2 & -1.17 & -1.11 & 0.32 & 4.17 \\ 
  flow3 & -1.30 & -1.29 & -0.70 & 0.40 \\ 
  flow4 & -0.68 & -0.72 & -0.38 & 0.53 \\ 
   \hline
\end{tabular}
\caption{D28 din} 
\end{table}
% latex table generated in R 3.2.3 by xtable 1.8-2 package
% Sun Aug  7 14:13:27 2016
\begin{table}[H]
\centering
\begin{tabular}{rrrrr}
  \hline
 & GAM\_avg & WRTDS\_avg & GAM\_pc & WRTDS\_pc \\ 
  \hline
overall & -3.30 & -3.34 & -0.26 & 0.21 \\ 
  annual1 & -3.19 & -3.22 & -0.41 & -0.09 \\ 
  annual2 & -3.30 & -3.28 & -0.19 & 0.33 \\ 
  annual3 & -3.27 & -3.34 & -0.19 & 0.33 \\ 
  annual4 & -3.23 & -3.32 & -0.11 & 0.08 \\ 
  annual5 & -3.44 & -3.46 & -0.41 & -0.09 \\ 
  seasonal1 & -2.53 & -2.53 & -0.12 & 0.26 \\ 
  seasonal2 & -3.56 & -3.56 & -0.41 & -0.02 \\ 
  seasonal3 & -4.02 & -4.02 & -0.12 & 0.26 \\ 
  seasonal4 & -3.09 & -3.09 & -0.34 & -0.13 \\ 
  flow1 & -3.42 & -3.49 & -0.41 & -0.02 \\ 
  flow2 & -3.46 & -3.45 & 0.21 & 0.64 \\ 
  flow3 & -3.54 & -3.55 & -0.38 & -0.09 \\ 
  flow4 & -2.78 & -2.87 & -0.41 & -0.02 \\ 
   \hline
\end{tabular}
\caption{D28 nh} 
\end{table}
% latex table generated in R 3.2.3 by xtable 1.8-2 package
% Sun Aug  7 14:13:27 2016
\begin{table}[H]
\centering
\begin{tabular}{rrrrr}
  \hline
 & GAM\_avg & WRTDS\_avg & GAM\_pc & WRTDS\_pc \\ 
  \hline
overall & -1.17 & -1.18 & -0.33 & 2.69 \\ 
  annual1 & -1.11 & -1.11 & -0.69 & 0.97 \\ 
  annual2 & -1.34 & -1.23 & -0.42 & 1.05 \\ 
  annual3 & -1.04 & -1.08 & -0.40 & 0.29 \\ 
  annual4 & -1.01 & -1.05 & -0.69 & 0.97 \\ 
  annual5 & -1.30 & -1.34 & -0.69 & 0.97 \\ 
  seasonal1 & -0.51 & -0.51 & 0.61 & 4.42 \\ 
  seasonal2 & -1.21 & -1.21 & 1.04 & 5.36 \\ 
  seasonal3 & -1.91 & -1.91 & 0.61 & 4.42 \\ 
  seasonal4 & -1.06 & -1.06 & 0.71 & 5.13 \\ 
  flow1 & -1.12 & -1.17 & 0.71 & 5.13 \\ 
  flow2 & -1.32 & -1.26 & 0.61 & 4.42 \\ 
  flow3 & -1.44 & -1.43 & 0.71 & 5.13 \\ 
  flow4 & -0.81 & -0.86 & 0.71 & 5.13 \\ 
   \hline
\end{tabular}
\caption{D28 no23} 
\end{table}
% latex table generated in R 3.2.3 by xtable 1.8-2 package
% Sun Aug  7 17:58:44 2016
\begin{table}[H]
\centering
\begin{tabular}{rrrrr}
  \hline
 & GAM\_avg & WRTDS\_avg & GAM\_pc & WRTDS\_pc \\ 
  \hline
overall & 0.21 & 0.13 & -1.57 & -1.97 \\ 
  annual1 & 0.10 & 0.06 & -1.37 & -1.97 \\ 
  annual2 & 0.23 & 0.06 & -1.63 & -2.70 \\ 
  annual3 & 0.28 & 0.28 & 0.03 & -0.60 \\ 
  annual4 & 0.37 & 0.28 & 3.35 & -17.10 \\ 
  annual5 & 0.08 & 0.02 & -3.38 & 4.06 \\ 
  seasonal1 & 0.30 & 0.30 & -1.08 & -1.34 \\ 
  seasonal2 & -0.15 & -0.15 & 14.18 & 6.76 \\ 
  seasonal3 & 0.29 & 0.29 & -0.99 & -1.76 \\ 
  seasonal4 & 0.39 & 0.39 & -0.61 & -0.48 \\ 
  flow1 & 0.44 & 0.43 & -0.30 & -0.26 \\ 
  flow2 & 0.59 & 0.44 & -0.39 & 0.31 \\ 
  flow3 & 0.35 & 0.21 & -2.24 & -3.98 \\ 
  flow4 & -0.53 & -0.53 & 0.55 & 0.44 \\ 
   \hline
\end{tabular}
\caption{C10 din} 
\end{table}
% latex table generated in R 3.2.3 by xtable 1.8-2 package
% Sun Aug  7 17:58:44 2016
\begin{table}[H]
\centering
\begin{tabular}{rrrrr}
  \hline
 & GAM\_avg & WRTDS\_avg & GAM\_pc & WRTDS\_pc \\ 
  \hline
overall & -3.10 & -3.36 & 0.97 & 1.24 \\ 
  annual1 & -2.22 & -2.24 & 0.25 & 1.24 \\ 
  annual2 & -2.59 & -2.77 & -0.09 & 0.01 \\ 
  annual3 & -3.01 & -3.23 & -0.16 & -0.05 \\ 
  annual4 & -3.47 & -3.75 & 0.10 & 0.18 \\ 
  annual5 & -3.94 & -4.42 & 0.03 & 0.06 \\ 
  seasonal1 & -2.34 & -2.34 & 1.27 & 1.28 \\ 
  seasonal2 & -3.65 & -3.65 & 0.41 & 0.69 \\ 
  seasonal3 & -3.69 & -3.69 & 0.77 & 1.28 \\ 
  seasonal4 & -2.71 & -2.71 & 2.46 & 2.12 \\ 
  flow1 & -3.37 & -3.67 & 1.43 & 1.58 \\ 
  flow2 & -3.05 & -3.35 & 0.69 & 0.64 \\ 
  flow3 & -2.91 & -3.20 & 1.84 & 1.85 \\ 
  flow4 & -3.07 & -3.22 & 0.50 & 0.76 \\ 
   \hline
\end{tabular}
\caption{C10 nh} 
\end{table}
% latex table generated in R 3.2.3 by xtable 1.8-2 package
% Sun Aug  7 17:58:44 2016
\begin{table}[H]
\centering
\begin{tabular}{rrrrr}
  \hline
 & GAM\_avg & WRTDS\_avg & GAM\_pc & WRTDS\_pc \\ 
  \hline
overall & 0.14 & 0.06 & -2.15 & -3.49 \\ 
  annual1 & -0.04 & -0.08 & -2.32 & -3.49 \\ 
  annual2 & 0.15 & -0.03 & -1.47 & -1.93 \\ 
  annual3 & 0.21 & 0.22 & -0.13 & -0.75 \\ 
  annual4 & 0.34 & 0.24 & 3.26 & -9.67 \\ 
  annual5 & 0.06 & -0.01 & -6.51 & 2.05 \\ 
  seasonal1 & 0.19 & 0.19 & -1.23 & -1.79 \\ 
  seasonal2 & -0.20 & -0.20 & 6.75 & 3.80 \\ 
  seasonal3 & 0.27 & 0.27 & -1.03 & -2.25 \\ 
  seasonal4 & 0.30 & 0.30 & -0.45 & 0.01 \\ 
  flow1 & 0.39 & 0.38 & -0.11 & -0.06 \\ 
  flow2 & 0.54 & 0.38 & -0.30 & 3.02 \\ 
  flow3 & 0.29 & 0.14 & 4.23 & 0.85 \\ 
  flow4 & -0.65 & -0.64 & 0.30 & 0.28 \\ 
   \hline
\end{tabular}
\caption{C10 no23} 
\end{table}
% latex table generated in R 3.2.3 by xtable 1.8-2 package
% Sun Aug  7 17:58:44 2016
\begin{table}[H]
\centering
\begin{tabular}{rrrrr}
  \hline
 & GAM\_avg & WRTDS\_avg & GAM\_pc & WRTDS\_pc \\ 
  \hline
overall & -1.00 & -1.03 & -0.19 & -0.12 \\ 
  annual1 & -1.21 & -1.26 & 0.13 & -0.12 \\ 
  annual2 & -1.14 & -1.13 & -0.44 & -0.30 \\ 
  annual3 & -0.96 & -0.94 & 0.54 & 0.33 \\ 
  annual4 & -0.91 & -0.97 & -0.36 & -0.33 \\ 
  annual5 & -0.84 & -0.89 & -0.02 & -0.03 \\ 
  seasonal1 & -0.98 & -0.98 & -0.29 & -0.18 \\ 
  seasonal2 & -1.08 & -1.08 & -0.16 & -0.11 \\ 
  seasonal3 & -1.18 & -1.18 & -0.11 & -0.05 \\ 
  seasonal4 & -0.75 & -0.75 & -0.24 & -0.18 \\ 
  flow1 & -0.71 & -0.75 & -0.54 & -0.47 \\ 
  flow2 & -0.88 & -0.93 & -0.42 & -0.35 \\ 
  flow3 & -1.06 & -1.10 & -0.19 & -0.14 \\ 
  flow4 & -1.34 & -1.33 & 0.06 & 0.01 \\ 
   \hline
\end{tabular}
\caption{C3 din} 
\end{table}
% latex table generated in R 3.2.3 by xtable 1.8-2 package
% Sun Aug  7 17:58:44 2016
\begin{table}[H]
\centering
\begin{tabular}{rrrrr}
  \hline
 & GAM\_avg & WRTDS\_avg & GAM\_pc & WRTDS\_pc \\ 
  \hline
overall & -1.61 & -1.65 & -0.20 & -0.08 \\ 
  annual1 & -1.96 & -1.99 & 0.18 & -0.08 \\ 
  annual2 & -1.74 & -1.78 & -0.49 & -0.33 \\ 
  annual3 & -1.59 & -1.52 & 0.55 & 0.37 \\ 
  annual4 & -1.43 & -1.57 & -0.34 & -0.31 \\ 
  annual5 & -1.41 & -1.48 & -0.08 & -0.06 \\ 
  seasonal1 & -1.93 & -1.93 & -0.26 & -0.10 \\ 
  seasonal2 & -1.64 & -1.64 & -0.24 & -0.09 \\ 
  seasonal3 & -1.57 & -1.57 & -0.14 & -0.05 \\ 
  seasonal4 & -1.32 & -1.32 & -0.11 & -0.11 \\ 
  flow1 & -1.11 & -1.16 & -0.46 & -0.34 \\ 
  flow2 & -1.29 & -1.44 & -0.46 & -0.33 \\ 
  flow3 & -1.60 & -1.67 & -0.22 & -0.15 \\ 
  flow4 & -2.42 & -2.33 & -0.09 & -0.09 \\ 
   \hline
\end{tabular}
\caption{C3 nh} 
\end{table}
% latex table generated in R 3.2.3 by xtable 1.8-2 package
% Sun Aug  7 17:58:44 2016
\begin{table}[H]
\centering
\begin{tabular}{rrrrr}
  \hline
 & GAM\_avg & WRTDS\_avg & GAM\_pc & WRTDS\_pc \\ 
  \hline
overall & -1.96 & -1.99 & -0.03 & -0.05 \\ 
  annual1 & -1.97 & -2.04 & -0.04 & -0.05 \\ 
  annual2 & -2.11 & -2.07 & -0.05 & -0.02 \\ 
  annual3 & -1.97 & -2.03 & 0.01 & -0.01 \\ 
  annual4 & -2.01 & -1.98 & -0.05 & -0.08 \\ 
  annual5 & -1.80 & -1.86 & 0.05 & 0.00 \\ 
  seasonal1 & -1.59 & -1.59 & -0.05 & -0.09 \\ 
  seasonal2 & -2.07 & -2.07 & -0.04 & -0.06 \\ 
  seasonal3 & -2.39 & -2.39 & 0.02 & 0.00 \\ 
  seasonal4 & -1.78 & -1.78 & -0.06 & -0.07 \\ 
  flow1 & -1.95 & -1.96 & -0.22 & -0.21 \\ 
  flow2 & -2.00 & -1.99 & -0.07 & -0.08 \\ 
  flow3 & -2.07 & -2.09 & 0.11 & -0.00 \\ 
  flow4 & -1.82 & -1.90 & 0.23 & 0.09 \\ 
   \hline
\end{tabular}
\caption{C3 no23} 
\end{table}
% latex table generated in R 3.2.3 by xtable 1.8-2 package
% Sun Aug  7 17:58:44 2016
\begin{table}[H]
\centering
\begin{tabular}{rrrrr}
  \hline
 & GAM\_avg & WRTDS\_avg & GAM\_pc & WRTDS\_pc \\ 
  \hline
overall & 0.31 & 0.23 & 1.26 & 63.01 \\ 
  annual1 & -0.04 & -0.08 & -2.50 & 63.01 \\ 
  annual2 & 0.23 & 0.08 & -1.43 & -2.89 \\ 
  annual3 & 0.33 & 0.33 & -0.02 & -0.37 \\ 
  annual4 & 0.52 & 0.44 & 1.75 & 4.74 \\ 
  annual5 & 0.42 & 0.34 & -0.24 & -0.21 \\ 
  seasonal1 & 0.53 & 0.53 & 0.08 & -0.00 \\ 
  seasonal2 & 0.04 & 0.04 & -0.07 & -0.04 \\ 
  seasonal3 & 0.07 & 0.07 & -0.78 & -0.78 \\ 
  seasonal4 & 0.58 & 0.58 & 0.85 & 2.36 \\ 
  flow1 & 0.33 & 0.34 & 0.91 & 2.92 \\ 
  flow2 & 0.66 & 0.54 & 1.43 & 4.55 \\ 
  flow3 & 0.56 & 0.39 & 1.09 & 12.59 \\ 
  flow4 & -0.32 &  & -0.44 & -0.20 \\ 
   \hline
\end{tabular}
\caption{P8 din} 
\end{table}
% latex table generated in R 3.2.3 by xtable 1.8-2 package
% Sun Aug  7 17:58:44 2016
\begin{table}[H]
\centering
\begin{tabular}{rrrrr}
  \hline
 & GAM\_avg & WRTDS\_avg & GAM\_pc & WRTDS\_pc \\ 
  \hline
overall & -1.89 & -2.00 & 0.17 & 0.22 \\ 
  annual1 & -2.13 & -2.24 & -0.19 & 0.22 \\ 
  annual2 & -1.73 & -1.84 & -0.31 & -0.24 \\ 
  annual3 & -1.67 & -1.70 & 0.03 & 0.07 \\ 
  annual4 & -1.48 & -1.65 & -0.35 & -0.22 \\ 
  annual5 & -2.26 & -2.40 & 0.83 & 0.51 \\ 
  seasonal1 & -1.25 & -1.25 & 0.64 & 0.61 \\ 
  seasonal2 & -2.33 & -2.33 & 0.08 & 0.16 \\ 
  seasonal3 & -2.59 & -2.59 & -0.04 & 0.00 \\ 
  seasonal4 & -1.37 & -1.37 & 0.28 & 0.29 \\ 
  flow1 & -2.32 & -2.28 & 0.09 & 0.18 \\ 
  flow2 & -1.56 & -1.73 & 0.24 & 0.18 \\ 
  flow3 & -1.40 & -1.69 & 0.67 & 0.41 \\ 
  flow4 & -2.25 &  & 0.41 & 0.46 \\ 
   \hline
\end{tabular}
\caption{P8 nh} 
\end{table}
% latex table generated in R 3.2.3 by xtable 1.8-2 package
% Sun Aug  7 17:58:44 2016
\begin{table}[H]
\centering
\begin{tabular}{rrrrr}
  \hline
 & GAM\_avg & WRTDS\_avg & GAM\_pc & WRTDS\_pc \\ 
  \hline
overall & 0.12 & 0.05 & -5.36 & -1.87 \\ 
  annual1 & -0.22 & -0.25 & 7.36 & -1.87 \\ 
  annual2 & 0.03 & -0.13 & -1.17 & -1.45 \\ 
  annual3 & 0.09 & 0.10 & -0.09 & -0.88 \\ 
  annual4 & 0.30 & 0.23 & 6.33 & -7.29 \\ 
  annual5 & 0.30 & 0.23 & 2.04 & 16.93 \\ 
  seasonal1 & 0.25 & 0.25 & 0.52 & 0.59 \\ 
  seasonal2 & -0.11 & -0.11 & -0.19 & -0.17 \\ 
  seasonal3 & -0.03 & -0.03 & -0.76 & -0.75 \\ 
  seasonal4 & 0.36 & 0.36 & 2.82 & -339.89 \\ 
  flow1 & 0.22 & 0.22 & 3.09 & -17.59 \\ 
  flow2 & 0.45 & 0.36 & 4.13 & -16.90 \\ 
  flow3 & 0.33 & 0.18 & 4.66 & -3.49 \\ 
  flow4 & -0.52 &  & -0.55 & -0.41 \\ 
   \hline
\end{tabular}
\caption{P8 no23} 
\end{table}
% latex table generated in R 3.2.3 by xtable 1.8-2 package
% Sun Aug  7 17:58:44 2016
\begin{table}[H]
\centering
\begin{tabular}{rrrrr}
  \hline
 & GAM\_avg & WRTDS\_avg & GAM\_pc & WRTDS\_pc \\ 
  \hline
overall & -1.02 & -1.01 & 0.18 & 0.16 \\ 
  annual1 & -1.04 & -1.04 & 0.18 & 0.16 \\ 
  annual2 & -1.14 & -1.04 & 0.03 & 0.01 \\ 
  annual3 & -0.89 & -0.93 & 0.40 & 0.13 \\ 
  annual4 &  &  &  &  \\ 
  annual5 & -1.00 & -1.03 & 0.15 & 0.07 \\ 
  seasonal1 & -0.46 & -0.46 & 0.02 & 0.13 \\ 
  seasonal2 & -1.19 & -1.19 & 0.12 & 0.09 \\ 
  seasonal3 & -1.57 & -1.57 & 0.26 & 0.09 \\ 
  seasonal4 & -0.85 & -0.85 & 0.32 & 0.14 \\ 
  flow1 & -0.87 & -0.92 & -0.11 & -0.11 \\ 
  flow2 & -1.13 & -1.07 & -0.18 & -0.13 \\ 
  flow3 & -1.26 & -1.23 & -0.07 & 0.40 \\ 
  flow4 & -0.81 & -0.84 & 3.35 & 1.57 \\ 
   \hline
\end{tabular}
\caption{D19 din} 
\end{table}
% latex table generated in R 3.2.3 by xtable 1.8-2 package
% Sun Aug  7 17:58:44 2016
\begin{table}[H]
\centering
\begin{tabular}{rrrrr}
  \hline
 & GAM\_avg & WRTDS\_avg & GAM\_pc & WRTDS\_pc \\ 
  \hline
overall & -3.12 & -3.14 & 0.05 & 0.06 \\ 
  annual1 & -3.12 & -3.13 & -0.01 & 0.06 \\ 
  annual2 & -3.09 & -3.05 & 0.03 & 0.03 \\ 
  annual3 & -3.02 & -3.06 & 0.05 & 0.02 \\ 
  annual4 &  &  &  &  \\ 
  annual5 & -3.22 & -3.26 & 0.03 & 0.02 \\ 
  seasonal1 & -2.35 & -2.35 & 0.01 & 0.04 \\ 
  seasonal2 & -3.48 & -3.48 & 0.02 & 0.06 \\ 
  seasonal3 & -3.84 & -3.84 & 0.06 & 0.04 \\ 
  seasonal4 & -2.81 & -2.81 & 0.08 & 0.02 \\ 
  flow1 & -3.08 & -3.12 & -0.03 & -0.06 \\ 
  flow2 & -3.26 & -3.23 & -0.08 & -0.04 \\ 
  flow3 & -3.38 & -3.40 & 0.11 & 0.16 \\ 
  flow4 & -2.78 & -2.81 & 0.37 & 0.32 \\ 
   \hline
\end{tabular}
\caption{D19 nh} 
\end{table}
% latex table generated in R 3.2.3 by xtable 1.8-2 package
% Sun Aug  7 17:58:44 2016
\begin{table}[H]
\centering
\begin{tabular}{rrrrr}
  \hline
 & GAM\_avg & WRTDS\_avg & GAM\_pc & WRTDS\_pc \\ 
  \hline
overall & -1.19 & -1.18 & 0.16 & 0.09 \\ 
  annual1 & -1.21 & -1.21 & 0.22 & 0.09 \\ 
  annual2 & -1.38 & -1.25 & -0.02 & -0.01 \\ 
  annual3 & -1.03 & -1.08 & 0.27 & 0.10 \\ 
  annual4 &  &  &  &  \\ 
  annual5 & -1.13 & -1.16 & 0.12 & 0.07 \\ 
  seasonal1 & -0.65 & -0.65 & 0.16 & 0.01 \\ 
  seasonal2 & -1.35 & -1.35 & -0.00 & 0.04 \\ 
  seasonal3 & -1.70 & -1.70 & 0.27 & 0.14 \\ 
  seasonal4 & -1.03 & -1.03 & 0.26 & 0.11 \\ 
  flow1 & -1.03 & -1.08 & -0.07 & -0.10 \\ 
  flow2 & -1.31 & -1.24 & -0.15 & -0.16 \\ 
  flow3 & -1.41 & -1.37 & 0.06 & 0.39 \\ 
  flow4 & -1.00 & -1.02 & 2.16 & 1.10 \\ 
   \hline
\end{tabular}
\caption{D19 no23} 
\end{table}
% latex table generated in R 3.2.3 by xtable 1.8-2 package
% Sun Aug  7 17:58:44 2016
\begin{table}[H]
\centering
\begin{tabular}{rrrrr}
  \hline
 & GAM\_avg & WRTDS\_avg & GAM\_pc & WRTDS\_pc \\ 
  \hline
overall & -0.92 & -0.92 & 0.19 & 0.19 \\ 
  annual1 & -0.96 & -0.96 & 0.31 & 0.19 \\ 
  annual2 & -1.08 & -0.99 & -0.03 & -0.07 \\ 
  annual3 & -0.83 & -0.85 & 0.36 & 0.15 \\ 
  annual4 & -0.86 & -0.88 & -0.10 & -0.12 \\ 
  annual5 & -0.89 & -0.93 & 0.11 & 0.10 \\ 
  seasonal1 & -0.56 & -0.56 & 0.15 & 0.05 \\ 
  seasonal2 & -1.02 & -1.02 & 0.13 & 0.20 \\ 
  seasonal3 & -1.35 & -1.35 & 0.20 & 0.17 \\ 
  seasonal4 & -0.75 & -0.75 & 0.25 & 0.18 \\ 
  flow1 & -0.78 & -0.81 & -0.08 & -0.12 \\ 
  flow2 & -1.00 & -0.96 & -0.22 & -0.14 \\ 
  flow3 & -1.08 & -1.08 & 0.02 & 0.31 \\ 
  flow4 & -0.83 & -0.83 & 0.63 & 0.51 \\ 
   \hline
\end{tabular}
\caption{D26 din} 
\end{table}
% latex table generated in R 3.2.3 by xtable 1.8-2 package
% Sun Aug  7 17:58:44 2016
\begin{table}[H]
\centering
\begin{tabular}{rrrrr}
  \hline
 & GAM\_avg & WRTDS\_avg & GAM\_pc & WRTDS\_pc \\ 
  \hline
overall & -2.40 & -2.41 & -0.07 & -0.03 \\ 
  annual1 & -2.57 & -2.58 & -0.01 & -0.03 \\ 
  annual2 & -2.47 & -2.42 & -0.12 & -0.06 \\ 
  annual3 & -2.32 & -2.30 & 0.19 & 0.04 \\ 
  annual4 & -2.27 & -2.34 & -0.03 & -0.04 \\ 
  annual5 & -2.40 & -2.42 & 0.03 & 0.04 \\ 
  seasonal1 & -2.12 & -2.12 & -0.09 & -0.01 \\ 
  seasonal2 & -2.72 & -2.72 & -0.09 & -0.03 \\ 
  seasonal3 & -2.68 & -2.68 & -0.09 & -0.08 \\ 
  seasonal4 & -2.10 & -2.10 & -0.02 & -0.03 \\ 
  flow1 & -2.28 & -2.30 & -0.18 & -0.13 \\ 
  flow2 & -2.39 & -2.43 & -0.17 & -0.13 \\ 
  flow3 & -2.52 & -2.52 & -0.05 & 0.05 \\ 
  flow4 & -2.42 & -2.39 & 0.13 & 0.06 \\ 
   \hline
\end{tabular}
\caption{D26 nh} 
\end{table}
% latex table generated in R 3.2.3 by xtable 1.8-2 package
% Sun Aug  7 17:58:44 2016
\begin{table}[H]
\centering
\begin{tabular}{rrrrr}
  \hline
 & GAM\_avg & WRTDS\_avg & GAM\_pc & WRTDS\_pc \\ 
  \hline
overall & -1.20 & -1.20 & 0.23 & 0.18 \\ 
  annual1 & -1.21 & -1.20 & 0.30 & 0.18 \\ 
  annual2 & -1.39 & -1.29 & 0.04 & -0.04 \\ 
  annual3 & -1.13 & -1.15 & 0.18 & 0.11 \\ 
  annual4 & -1.16 & -1.16 & -0.08 & -0.08 \\ 
  annual5 & -1.15 & -1.20 & 0.10 & 0.08 \\ 
  seasonal1 & -0.82 & -0.82 & 0.23 & 0.06 \\ 
  seasonal2 & -1.24 & -1.24 & 0.22 & 0.25 \\ 
  seasonal3 & -1.68 & -1.68 & 0.26 & 0.22 \\ 
  seasonal4 & -1.07 & -1.07 & 0.21 & 0.14 \\ 
  flow1 & -1.06 & -1.10 & 0.04 & -0.05 \\ 
  flow2 & -1.31 & -1.25 & -0.14 & -0.11 \\ 
  flow3 & -1.38 & -1.37 & 0.04 & 0.28 \\ 
  flow4 & -1.06 & -1.08 & 0.51 & 0.45 \\ 
   \hline
\end{tabular}
\caption{D26 no23} 
\end{table}
% latex table generated in R 3.2.3 by xtable 1.8-2 package
% Sun Aug  7 17:58:44 2016
\begin{table}[H]
\centering
\begin{tabular}{rrrrr}
  \hline
 & GAM\_avg & WRTDS\_avg & GAM\_pc & WRTDS\_pc \\ 
  \hline
overall & -1.03 & -1.04 & 0.63 & 0.61 \\ 
  annual1 & -0.97 & -0.97 & 0.27 & 0.61 \\ 
  annual2 & -1.14 & -1.06 & 0.11 & 0.01 \\ 
  annual3 & -0.92 & -0.95 & 0.20 & 0.02 \\ 
  annual4 & -0.90 & -0.94 & 0.14 & 0.01 \\ 
  annual5 & -1.17 & -1.20 & 0.32 & 0.30 \\ 
  seasonal1 & -0.39 & -0.39 & 0.15 & 0.09 \\ 
  seasonal2 & -1.05 & -1.05 & 0.15 & 0.41 \\ 
  seasonal3 & -1.80 & -1.80 & 0.66 & 0.60 \\ 
  seasonal4 & -0.91 & -0.91 & 0.84 & 0.63 \\ 
  flow1 & -0.99 & -1.04 & 0.48 & 0.19 \\ 
  flow2 & -1.17 & -1.11 & 0.35 & 0.58 \\ 
  flow3 & -1.30 & -1.29 & 0.71 & 0.52 \\ 
  flow4 & -0.68 & -0.72 & 2.97 & 1.50 \\ 
   \hline
\end{tabular}
\caption{D28 din} 
\end{table}
% latex table generated in R 3.2.3 by xtable 1.8-2 package
% Sun Aug  7 17:58:44 2016
\begin{table}[H]
\centering
\begin{tabular}{rrrrr}
  \hline
 & GAM\_avg & WRTDS\_avg & GAM\_pc & WRTDS\_pc \\ 
  \hline
overall & -3.30 & -3.34 & 0.09 & 0.08 \\ 
  annual1 & -3.19 & -3.22 & 0.03 & 0.08 \\ 
  annual2 & -3.30 & -3.28 & 0.11 & 0.07 \\ 
  annual3 & -3.27 & -3.34 & 0.03 & -0.04 \\ 
  annual4 & -3.23 & -3.32 & 0.06 & 0.03 \\ 
  annual5 & -3.44 & -3.46 & -0.01 & 0.03 \\ 
  seasonal1 & -2.53 & -2.53 & 0.01 & 0.03 \\ 
  seasonal2 & -3.56 & -3.56 & 0.10 & 0.08 \\ 
  seasonal3 & -4.02 & -4.02 & 0.11 & 0.08 \\ 
  seasonal4 & -3.09 & -3.09 & 0.04 & 0.02 \\ 
  flow1 & -3.42 & -3.49 & -0.01 & -0.01 \\ 
  flow2 & -3.46 & -3.45 & 0.06 & 0.09 \\ 
  flow3 & -3.54 & -3.55 & 0.09 & 0.08 \\ 
  flow4 & -2.78 & -2.87 & 0.30 & 0.21 \\ 
   \hline
\end{tabular}
\caption{D28 nh} 
\end{table}
% latex table generated in R 3.2.3 by xtable 1.8-2 package
% Sun Aug  7 17:58:44 2016
\begin{table}[H]
\centering
\begin{tabular}{rrrrr}
  \hline
 & GAM\_avg & WRTDS\_avg & GAM\_pc & WRTDS\_pc \\ 
  \hline
overall & -1.17 & -1.18 & 0.58 & 0.51 \\ 
  annual1 & -1.11 & -1.11 & 0.27 & 0.51 \\ 
  annual2 & -1.34 & -1.23 & 0.03 & -0.00 \\ 
  annual3 & -1.04 & -1.08 & 0.17 & 0.01 \\ 
  annual4 & -1.01 & -1.05 & 0.12 & 0.01 \\ 
  annual5 & -1.30 & -1.34 & 0.34 & 0.31 \\ 
  seasonal1 & -0.51 & -0.51 & 0.30 & 0.18 \\ 
  seasonal2 & -1.21 & -1.21 & 0.13 & 0.28 \\ 
  seasonal3 & -1.91 & -1.91 & 0.80 & 0.66 \\ 
  seasonal4 & -1.06 & -1.06 & 0.96 & 0.69 \\ 
  flow1 & -1.12 & -1.17 & 0.34 & 0.17 \\ 
  flow2 & -1.32 & -1.26 & 0.02 & 0.23 \\ 
  flow3 & -1.44 & -1.43 & 0.81 & 0.51 \\ 
  flow4 & -0.81 & -0.86 & 2.40 & 1.33 \\ 
   \hline
\end{tabular}
\caption{D28 no23} 
\end{table}


\section{Like Table 5 in Beck and Murphy EMA}


% latex table generated in R 3.2.3 by xtable 1.8-2 package
% Sun Aug  7 14:46:44 2016
\begin{table}[H]
\centering
\begin{tabular}{rrr}
  \hline
 & Avg Diff & RMSE \\ 
  \hline
overall & -36.56 & 5.09 \\ 
  annual1 & -45.73 & 2.10 \\ 
  annual2 & -74.74 & 2.05 \\ 
  annual3 & -2.66 & 2.57 \\ 
  annual4 & -25.01 & 1.77 \\ 
  annual5 & -76.83 & 2.75 \\ 
  seasonal1 & -18.77 & 2.82 \\ 
  seasonal2 & 31.39 & 2.44 \\ 
  seasonal3 & -50.15 & 2.67 \\ 
  seasonal4 & -13.55 & 2.21 \\ 
  flow1 & -4.14 & 2.26 \\ 
  flow2 & -25.92 & 2.32 \\ 
  flow3 & -39.93 & 2.64 \\ 
  flow4 & -0.74 & 2.91 \\ 
   \hline
\end{tabular}
\caption{C10 din} 
\end{table}
% latex table generated in R 3.2.3 by xtable 1.8-2 package
% Sun Aug  7 14:46:44 2016
\begin{table}[H]
\centering
\begin{tabular}{rrr}
  \hline
 & Avg Diff & RMSE \\ 
  \hline
overall & 7.84 & 10.29 \\ 
  annual1 & -1.56 & 2.57 \\ 
  annual2 & 6.82 & 3.02 \\ 
  annual3 & 7.21 & 3.93 \\ 
  annual4 & 8.29 & 4.50 \\ 
  annual5 & 12.37 & 7.38 \\ 
  seasonal1 & 3.47 & 3.31 \\ 
  seasonal2 & 7.45 & 5.22 \\ 
  seasonal3 & 13.25 & 7.65 \\ 
  seasonal4 & 4.72 & 3.03 \\ 
  flow1 & 6.92 & 6.55 \\ 
  flow2 & 9.95 & 4.79 \\ 
  flow3 & 9.70 & 4.48 \\ 
  flow4 & 5.01 & 4.48 \\ 
   \hline
\end{tabular}
\caption{C10 nh} 
\end{table}
% latex table generated in R 3.2.3 by xtable 1.8-2 package
% Sun Aug  7 14:46:44 2016
\begin{table}[H]
\centering
\begin{tabular}{rrr}
  \hline
 & Avg Diff & RMSE \\ 
  \hline
overall & -55.26 & 5.37 \\ 
  annual1 & 111.26 & 2.24 \\ 
  annual2 & -119.73 & 2.18 \\ 
  annual3 & 3.59 & 2.66 \\ 
  annual4 & -30.48 & 1.93 \\ 
  annual5 & -124.16 & 2.88 \\ 
  seasonal1 & -29.58 & 3.03 \\ 
  seasonal2 & 27.44 & 2.60 \\ 
  seasonal3 & -55.85 & 2.79 \\ 
  seasonal4 & -17.29 & 2.28 \\ 
  flow1 & -4.47 & 2.31 \\ 
  flow2 & -29.03 & 2.47 \\ 
  flow3 & -50.99 & 2.77 \\ 
  flow4 & -1.40 & 3.12 \\ 
   \hline
\end{tabular}
\caption{C10 no23} 
\end{table}
% latex table generated in R 3.2.3 by xtable 1.8-2 package
% Sun Aug  7 14:46:44 2016
\begin{table}[H]
\centering
\begin{tabular}{rrr}
  \hline
 & Avg Diff & RMSE \\ 
  \hline
overall & 2.78 & 4.03 \\ 
  annual1 & 3.52 & 1.33 \\ 
  annual2 & -0.91 & 1.54 \\ 
  annual3 & -1.17 & 1.85 \\ 
  annual4 & 6.25 & 1.41 \\ 
  annual5 & 6.09 & 2.59 \\ 
  seasonal1 & 1.73 & 2.70 \\ 
  seasonal2 & 4.38 & 2.23 \\ 
  seasonal3 & 2.38 & 1.28 \\ 
  seasonal4 & 2.40 & 1.54 \\ 
  flow1 & 4.53 & 1.32 \\ 
  flow2 & 6.43 & 1.62 \\ 
  flow3 & 3.55 & 1.98 \\ 
  flow4 & -1.06 & 2.82 \\ 
   \hline
\end{tabular}
\caption{C3 din} 
\end{table}
% latex table generated in R 3.2.3 by xtable 1.8-2 package
% Sun Aug  7 14:46:44 2016
\begin{table}[H]
\centering
\begin{tabular}{rrr}
  \hline
 & Avg Diff & RMSE \\ 
  \hline
overall & 2.50 & 7.62 \\ 
  annual1 & 1.75 & 2.68 \\ 
  annual2 & 2.09 & 2.98 \\ 
  annual3 & -4.75 & 4.32 \\ 
  annual4 & 9.37 & 2.44 \\ 
  annual5 & 4.45 & 4.17 \\ 
  seasonal1 & 1.00 & 5.09 \\ 
  seasonal2 & 5.35 & 3.64 \\ 
  seasonal3 & 3.47 & 1.97 \\ 
  seasonal4 & -0.09 & 3.87 \\ 
  flow1 & 4.15 & 3.27 \\ 
  flow2 & 11.27 & 2.67 \\ 
  flow3 & 4.15 & 3.45 \\ 
  flow4 & -3.87 & 5.32 \\ 
   \hline
\end{tabular}
\caption{C3 nh} 
\end{table}
% latex table generated in R 3.2.3 by xtable 1.8-2 package
% Sun Aug  7 14:46:44 2016
\begin{table}[H]
\centering
\begin{tabular}{rrr}
  \hline
 & Avg Diff & RMSE \\ 
  \hline
overall & 1.38 & 3.74 \\ 
  annual1 & 3.88 & 1.95 \\ 
  annual2 & -1.81 & 1.27 \\ 
  annual3 & 3.06 & 1.74 \\ 
  annual4 & -1.75 & 1.21 \\ 
  annual5 & 3.27 & 2.02 \\ 
  seasonal1 & 1.55 & 2.40 \\ 
  seasonal2 & 1.54 & 1.74 \\ 
  seasonal3 & 0.22 & 1.43 \\ 
  seasonal4 & 2.64 & 1.79 \\ 
  flow1 & 0.54 & 1.65 \\ 
  flow2 & -0.29 & 1.56 \\ 
  flow3 & 1.05 & 1.79 \\ 
  flow4 & 4.42 & 2.38 \\ 
   \hline
\end{tabular}
\caption{C3 no23} 
\end{table}
% latex table generated in R 3.2.3 by xtable 1.8-2 package
% Sun Aug  7 14:46:44 2016
\begin{table}[H]
\centering
\begin{tabular}{rrr}
  \hline
 & Avg Diff & RMSE \\ 
  \hline
overall & -23.83 & 4.95 \\ 
  annual1 & 83.35 & 2.05 \\ 
  annual2 & -65.51 & 2.22 \\ 
  annual3 & -1.18 & 2.20 \\ 
  annual4 & -14.96 & 1.99 \\ 
  annual5 & -20.06 & 2.57 \\ 
  seasonal1 & -5.31 & 2.91 \\ 
  seasonal2 & -243.75 & 2.54 \\ 
  seasonal3 & -109.39 & 2.38 \\ 
  seasonal4 & -13.77 & 1.99 \\ 
  flow1 & 2.77 & 1.54 \\ 
  flow2 & -17.19 & 2.19 \\ 
  flow3 & -31.06 & 2.63 \\ 
  flow4 & 4.03 & 3.23 \\ 
   \hline
\end{tabular}
\caption{P8 din} 
\end{table}
% latex table generated in R 3.2.3 by xtable 1.8-2 package
% Sun Aug  7 14:46:44 2016
\begin{table}[H]
\centering
\begin{tabular}{rrr}
  \hline
 & Avg Diff & RMSE \\ 
  \hline
overall & 5.56 & 6.85 \\ 
  annual1 & 3.59 & 2.80 \\ 
  annual2 & 5.19 & 2.41 \\ 
  annual3 & 2.11 & 2.59 \\ 
  annual4 & 11.49 & 2.87 \\ 
  annual5 & 6.05 & 4.28 \\ 
  seasonal1 & 2.55 & 3.63 \\ 
  seasonal2 & 6.60 & 3.58 \\ 
  seasonal3 & 3.83 & 3.47 \\ 
  seasonal4 & 9.89 & 2.98 \\ 
  flow1 & -2.61 & 2.90 \\ 
  flow2 & 10.68 & 2.94 \\ 
  flow3 & 20.42 & 3.90 \\ 
  flow4 & 1.19 & 3.83 \\ 
   \hline
\end{tabular}
\caption{P8 nh} 
\end{table}
% latex table generated in R 3.2.3 by xtable 1.8-2 package
% Sun Aug  7 14:46:44 2016
\begin{table}[H]
\centering
\begin{tabular}{rrr}
  \hline
 & Avg Diff & RMSE \\ 
  \hline
overall & -55.53 & 4.83 \\ 
  annual1 & 12.43 & 1.99 \\ 
  annual2 & -531.61 & 2.30 \\ 
  annual3 & 20.28 & 1.97 \\ 
  annual4 & -23.25 & 1.84 \\ 
  annual5 & -25.12 & 2.62 \\ 
  seasonal1 & -5.12 & 2.88 \\ 
  seasonal2 & 89.31 & 2.52 \\ 
  seasonal3 & 258.74 & 2.28 \\ 
  seasonal4 & -20.68 & 1.87 \\ 
  flow1 & -0.28 & 1.48 \\ 
  flow2 & -19.42 & 1.99 \\ 
  flow3 & -46.59 & 2.64 \\ 
  flow4 & 2.97 & 3.20 \\ 
   \hline
\end{tabular}
\caption{P8 no23} 
\end{table}
% latex table generated in R 3.2.3 by xtable 1.8-2 package
% Sun Aug  7 14:46:44 2016
\begin{table}[H]
\centering
\begin{tabular}{rrr}
  \hline
 & Avg Diff & RMSE \\ 
  \hline
overall & -0.30 & 4.22 \\ 
  annual1 & 0.02 & 2.90 \\ 
  annual2 & -8.62 & 1.58 \\ 
  annual3 & 5.04 & 1.80 \\ 
  annual4 &  & 0.00 \\ 
  annual5 & 3.07 & 1.91 \\ 
  seasonal1 & 2.59 & 2.50 \\ 
  seasonal2 & -6.50 & 2.40 \\ 
  seasonal3 & 0.87 & 2.04 \\ 
  seasonal4 & 4.88 & 1.27 \\ 
  flow1 & 6.12 & 1.92 \\ 
  flow2 & -5.24 & 2.29 \\ 
  flow3 & -2.35 & 2.04 \\ 
  flow4 & 2.97 & 2.17 \\ 
   \hline
\end{tabular}
\caption{D19 din} 
\end{table}
% latex table generated in R 3.2.3 by xtable 1.8-2 package
% Sun Aug  7 14:46:44 2016
\begin{table}[H]
\centering
\begin{tabular}{rrr}
  \hline
 & Avg Diff & RMSE \\ 
  \hline
overall & 0.51 & 3.11 \\ 
  annual1 & 0.41 & 1.69 \\ 
  annual2 & -1.18 & 1.22 \\ 
  annual3 & 1.33 & 1.43 \\ 
  annual4 &  & 0.00 \\ 
  annual5 & 1.22 & 1.81 \\ 
  seasonal1 & 0.08 & 1.33 \\ 
  seasonal2 & -0.30 & 1.97 \\ 
  seasonal3 & 1.42 & 1.31 \\ 
  seasonal4 & 0.66 & 1.52 \\ 
  flow1 & 1.31 & 1.76 \\ 
  flow2 & -0.91 & 1.60 \\ 
  flow3 & 0.71 & 1.42 \\ 
  flow4 & 1.07 & 1.40 \\ 
   \hline
\end{tabular}
\caption{D19 nh} 
\end{table}
% latex table generated in R 3.2.3 by xtable 1.8-2 package
% Sun Aug  7 14:46:44 2016
\begin{table}[H]
\centering
\begin{tabular}{rrr}
  \hline
 & Avg Diff & RMSE \\ 
  \hline
overall & -0.83 & 4.50 \\ 
  annual1 & -0.65 & 3.18 \\ 
  annual2 & -8.95 & 1.83 \\ 
  annual3 & 4.92 & 1.54 \\ 
  annual4 &  & 0.00 \\ 
  annual5 & 2.85 & 2.11 \\ 
  seasonal1 & 2.47 & 2.69 \\ 
  seasonal2 & -6.96 & 2.49 \\ 
  seasonal3 & 0.48 & 2.24 \\ 
  seasonal4 & 3.04 & 1.34 \\ 
  flow1 & 3.92 & 1.94 \\ 
  flow2 & -5.01 & 2.50 \\ 
  flow3 & -2.84 & 2.05 \\ 
  flow4 & 2.55 & 2.45 \\ 
   \hline
\end{tabular}
\caption{D19 no23} 
\end{table}
% latex table generated in R 3.2.3 by xtable 1.8-2 package
% Sun Aug  7 14:46:44 2016
\begin{table}[H]
\centering
\begin{tabular}{rrr}
  \hline
 & Avg Diff & RMSE \\ 
  \hline
overall & 0.01 & 3.26 \\ 
  annual1 & -0.19 & 1.84 \\ 
  annual2 & -8.56 & 1.34 \\ 
  annual3 & 1.92 & 1.29 \\ 
  annual4 & 2.06 & 0.98 \\ 
  annual5 & 4.80 & 1.67 \\ 
  seasonal1 & -1.36 & 1.99 \\ 
  seasonal2 & -2.03 & 1.77 \\ 
  seasonal3 & -0.16 & 1.40 \\ 
  seasonal4 & 4.12 & 1.26 \\ 
  flow1 & 4.89 & 1.60 \\ 
  flow2 & -3.98 & 1.55 \\ 
  flow3 & -0.40 & 1.50 \\ 
  flow4 & 0.88 & 1.84 \\ 
   \hline
\end{tabular}
\caption{D26 din} 
\end{table}
% latex table generated in R 3.2.3 by xtable 1.8-2 package
% Sun Aug  7 14:46:44 2016
\begin{table}[H]
\centering
\begin{tabular}{rrr}
  \hline
 & Avg Diff & RMSE \\ 
  \hline
overall & 0.29 & 3.45 \\ 
  annual1 & 0.46 & 1.26 \\ 
  annual2 & -2.11 & 1.29 \\ 
  annual3 & -1.08 & 2.05 \\ 
  annual4 & 3.15 & 1.31 \\ 
  annual5 & 0.98 & 1.67 \\ 
  seasonal1 & 0.49 & 1.93 \\ 
  seasonal2 & -0.38 & 1.92 \\ 
  seasonal3 & 1.12 & 1.52 \\ 
  seasonal4 & -0.11 & 1.48 \\ 
  flow1 & 0.71 & 1.87 \\ 
  flow2 & 1.55 & 1.64 \\ 
  flow3 & -0.05 & 1.55 \\ 
  flow4 & -1.01 & 1.82 \\ 
   \hline
\end{tabular}
\caption{D26 nh} 
\end{table}
% latex table generated in R 3.2.3 by xtable 1.8-2 package
% Sun Aug  7 14:46:44 2016
\begin{table}[H]
\centering
\begin{tabular}{rrr}
  \hline
 & Avg Diff & RMSE \\ 
  \hline
overall & -0.14 & 3.55 \\ 
  annual1 & -0.38 & 2.12 \\ 
  annual2 & -7.15 & 1.45 \\ 
  annual3 & 2.19 & 1.28 \\ 
  annual4 & 0.04 & 1.01 \\ 
  annual5 & 4.21 & 1.82 \\ 
  seasonal1 & -1.45 & 2.20 \\ 
  seasonal2 & -1.91 & 1.87 \\ 
  seasonal3 & -0.74 & 1.51 \\ 
  seasonal4 & 3.83 & 1.40 \\ 
  flow1 & 4.14 & 1.66 \\ 
  flow2 & -4.46 & 1.73 \\ 
  flow3 & -0.81 & 1.61 \\ 
  flow4 & 1.81 & 2.06 \\ 
   \hline
\end{tabular}
\caption{D26 no23} 
\end{table}
% latex table generated in R 3.2.3 by xtable 1.8-2 package
% Sun Aug  7 14:46:44 2016
\begin{table}[H]
\centering
\begin{tabular}{rrr}
  \hline
 & Avg Diff & RMSE \\ 
  \hline
overall & 0.59 & 4.82 \\ 
  annual1 & 0.10 & 2.92 \\ 
  annual2 & -7.72 & 1.78 \\ 
  annual3 & 3.03 & 1.91 \\ 
  annual4 & 4.57 & 1.66 \\ 
  annual5 & 3.05 & 2.26 \\ 
  seasonal1 & -4.63 & 2.71 \\ 
  seasonal2 & -2.13 & 2.57 \\ 
  seasonal3 & -1.52 & 2.30 \\ 
  seasonal4 & 9.93 & 1.99 \\ 
  flow1 & 5.57 & 2.47 \\ 
  flow2 & -5.10 & 2.77 \\ 
  flow3 & -1.37 & 1.85 \\ 
  flow4 & 6.67 & 2.45 \\ 
   \hline
\end{tabular}
\caption{D28 din} 
\end{table}
% latex table generated in R 3.2.3 by xtable 1.8-2 package
% Sun Aug  7 14:46:44 2016
\begin{table}[H]
\centering
\begin{tabular}{rrr}
  \hline
 & Avg Diff & RMSE \\ 
  \hline
overall & 1.15 & 5.80 \\ 
  annual1 & 1.00 & 3.13 \\ 
  annual2 & -0.35 & 2.24 \\ 
  annual3 & 1.93 & 2.73 \\ 
  annual4 & 2.97 & 2.38 \\ 
  annual5 & 0.55 & 2.41 \\ 
  seasonal1 & 0.69 & 3.21 \\ 
  seasonal2 & -0.11 & 3.38 \\ 
  seasonal3 & 2.04 & 2.48 \\ 
  seasonal4 & 1.80 & 2.40 \\ 
  flow1 & 2.00 & 2.75 \\ 
  flow2 & -0.34 & 2.96 \\ 
  flow3 & 0.13 & 2.36 \\ 
  flow4 & 3.21 & 3.43 \\ 
   \hline
\end{tabular}
\caption{D28 nh} 
\end{table}
% latex table generated in R 3.2.3 by xtable 1.8-2 package
% Sun Aug  7 14:46:44 2016
\begin{table}[H]
\centering
\begin{tabular}{rrr}
  \hline
 & Avg Diff & RMSE \\ 
  \hline
overall & 0.41 & 4.95 \\ 
  annual1 & 0.21 & 3.12 \\ 
  annual2 & -8.40 & 1.87 \\ 
  annual3 & 3.62 & 1.91 \\ 
  annual4 & 3.98 & 1.60 \\ 
  annual5 & 3.30 & 2.26 \\ 
  seasonal1 & -1.11 & 2.73 \\ 
  seasonal2 & -3.56 & 2.57 \\ 
  seasonal3 & -0.87 & 2.52 \\ 
  seasonal4 & 8.03 & 2.03 \\ 
  flow1 & 4.51 & 2.52 \\ 
  flow2 & -4.83 & 2.85 \\ 
  flow3 & -0.86 & 1.89 \\ 
  flow4 & 5.47 & 2.55 \\ 
   \hline
\end{tabular}
\caption{D28 no23} 
\end{table}


\section{Like Table 6 in Beck and Murphy EMA}


% latex table generated in R 3.2.3 by xtable 1.8-2 package
% Sun Aug  7 19:16:52 2016
\begin{table}[H]
\centering
\begin{tabular}{rrrrr}
  \hline
 & intercept & interceptSignificant & slope & slopeSignificant \\ 
  \hline
betas & -0.04 & 1.00 & 0.85 & 1.00 \\ 
  annualBetas1 & -0.03 & 0.00 & 0.81 & 1.00 \\ 
  annualBetas2 & -0.14 & 1.00 & 0.87 & 1.00 \\ 
  annualBetas3 & 0.09 & 1.00 & 0.65 & 1.00 \\ 
  annualBetas4 & -0.06 & 1.00 & 0.92 & 1.00 \\ 
  annualBetas5 & -0.06 & 1.00 & 0.89 & 1.00 \\ 
  seasonalBetas1 & -0.01 & 0.00 & 0.86 & 1.00 \\ 
  seasonalBetas2 & -0.07 & 1.00 & 0.85 & 1.00 \\ 
  seasonalBetas3 & -0.11 & 1.00 & 0.87 & 1.00 \\ 
  seasonalBetas4 & 0.06 & 1.00 & 0.71 & 1.00 \\ 
  flowBetas1 & 0.18 & 1.00 & 0.56 & 1.00 \\ 
  flowBetas2 & -0.07 & 0.00 & 0.85 & 0.00 \\ 
  flowBetas3 & -0.02 & 0.00 & 0.67 & 1.00 \\ 
  flowBetas4 & -0.04 & 0.00 & 0.91 & 0.00 \\ 
   \hline
\end{tabular}
\caption{C10 din} 
\end{table}
% latex table generated in R 3.2.3 by xtable 1.8-2 package
% Sun Aug  7 19:16:52 2016
\begin{table}[H]
\centering
\begin{tabular}{rrrrr}
  \hline
 & intercept & interceptSignificant & slope & slopeSignificant \\ 
  \hline
betas & 0.08 & 0.00 & 1.11 & 1.00 \\ 
  annualBetas1 & -0.38 & 1.00 & 0.84 & 1.00 \\ 
  annualBetas2 & -0.51 & 1.00 & 0.87 & 1.00 \\ 
  annualBetas3 & -0.35 & 1.00 & 0.95 & 0.00 \\ 
  annualBetas4 & 0.40 & 1.00 & 1.20 & 1.00 \\ 
  annualBetas5 & 1.00 & 1.00 & 1.38 & 1.00 \\ 
  seasonalBetas1 & -0.53 & 1.00 & 0.81 & 1.00 \\ 
  seasonalBetas2 & 0.36 & 0.00 & 1.17 & 1.00 \\ 
  seasonalBetas3 & 0.65 & 1.00 & 1.32 & 1.00 \\ 
  seasonalBetas4 & -0.29 & 1.00 & 0.94 & 1.00 \\ 
  flowBetas1 & 0.27 & 0.00 & 1.17 & 1.00 \\ 
  flowBetas2 & -0.15 & 0.00 & 1.05 & 0.00 \\ 
  flowBetas3 & 0.02 & 0.00 & 1.10 & 1.00 \\ 
  flowBetas4 & 0.09 & 0.00 & 1.08 & 0.00 \\ 
   \hline
\end{tabular}
\caption{C10 nh} 
\end{table}
% latex table generated in R 3.2.3 by xtable 1.8-2 package
% Sun Aug  7 19:16:52 2016
\begin{table}[H]
\centering
\begin{tabular}{rrrrr}
  \hline
 & intercept & interceptSignificant & slope & slopeSignificant \\ 
  \hline
betas & -0.06 & 1.00 & 0.85 & 1.00 \\ 
  annualBetas1 & -0.05 & 1.00 & 0.80 & 1.00 \\ 
  annualBetas2 & -0.16 & 1.00 & 0.87 & 1.00 \\ 
  annualBetas3 & 0.09 & 1.00 & 0.63 & 1.00 \\ 
  annualBetas4 & -0.08 & 1.00 & 0.92 & 1.00 \\ 
  annualBetas5 & -0.06 & 1.00 & 0.88 & 1.00 \\ 
  seasonalBetas1 & -0.03 & 0.00 & 0.86 & 1.00 \\ 
  seasonalBetas2 & -0.08 & 1.00 & 0.85 & 1.00 \\ 
  seasonalBetas3 & -0.11 & 1.00 & 0.86 & 1.00 \\ 
  seasonalBetas4 & 0.03 & 0.00 & 0.74 & 1.00 \\ 
  flowBetas1 & 0.18 & 1.00 & 0.51 & 1.00 \\ 
  flowBetas2 & -0.05 & 0.00 & 0.79 & 1.00 \\ 
  flowBetas3 & -0.04 & 0.00 & 0.64 & 1.00 \\ 
  flowBetas4 & -0.05 & 0.00 & 0.90 & 0.00 \\ 
   \hline
\end{tabular}
\caption{C10 no23} 
\end{table}
% latex table generated in R 3.2.3 by xtable 1.8-2 package
% Sun Aug  7 19:16:52 2016
\begin{table}[H]
\centering
\begin{tabular}{rrrrr}
  \hline
 & intercept & interceptSignificant & slope & slopeSignificant \\ 
  \hline
betas & -0.13 & 1.00 & 0.89 & 1.00 \\ 
  annualBetas1 & -0.13 & 0.00 & 0.93 & 0.00 \\ 
  annualBetas2 & -0.16 & 1.00 & 0.85 & 1.00 \\ 
  annualBetas3 & -0.17 & 1.00 & 0.81 & 1.00 \\ 
  annualBetas4 & -0.08 & 0.00 & 0.97 & 0.00 \\ 
  annualBetas5 & -0.14 & 1.00 & 0.90 & 0.00 \\ 
  seasonalBetas1 & -0.21 & 1.00 & 0.80 & 1.00 \\ 
  seasonalBetas2 & -0.12 & 1.00 & 0.93 & 0.00 \\ 
  seasonalBetas3 & -0.26 & 1.00 & 0.80 & 1.00 \\ 
  seasonalBetas4 & -0.14 & 1.00 & 0.84 & 1.00 \\ 
  flowBetas1 & -0.09 & 1.00 & 0.92 & 0.00 \\ 
  flowBetas2 & -0.17 & 1.00 & 0.87 & 1.00 \\ 
  flowBetas3 & -0.25 & 1.00 & 0.80 & 1.00 \\ 
  flowBetas4 & -0.11 & 0.00 & 0.90 & 0.00 \\ 
   \hline
\end{tabular}
\caption{C3 din} 
\end{table}
% latex table generated in R 3.2.3 by xtable 1.8-2 package
% Sun Aug  7 19:16:52 2016
\begin{table}[H]
\centering
\begin{tabular}{rrrrr}
  \hline
 & intercept & interceptSignificant & slope & slopeSignificant \\ 
  \hline
betas & -0.33 & 1.00 & 0.82 & 1.00 \\ 
  annualBetas1 & -0.25 & 0.00 & 0.89 & 0.00 \\ 
  annualBetas2 & -0.37 & 1.00 & 0.81 & 1.00 \\ 
  annualBetas3 & -0.60 & 1.00 & 0.58 & 1.00 \\ 
  annualBetas4 & -0.19 & 1.00 & 0.96 & 0.00 \\ 
  annualBetas5 & -0.22 & 1.00 & 0.89 & 0.00 \\ 
  seasonalBetas1 & -0.47 & 1.00 & 0.77 & 1.00 \\ 
  seasonalBetas2 & -0.22 & 1.00 & 0.92 & 0.00 \\ 
  seasonalBetas3 & -0.34 & 1.00 & 0.82 & 1.00 \\ 
  seasonalBetas4 & -0.51 & 1.00 & 0.61 & 1.00 \\ 
  flowBetas1 & -0.58 & 1.00 & 0.52 & 1.00 \\ 
  flowBetas2 & -0.37 & 1.00 & 0.83 & 1.00 \\ 
  flowBetas3 & -0.44 & 1.00 & 0.77 & 1.00 \\ 
  flowBetas4 & -0.26 & 0.00 & 0.85 & 0.00 \\ 
   \hline
\end{tabular}
\caption{C3 nh} 
\end{table}
% latex table generated in R 3.2.3 by xtable 1.8-2 package
% Sun Aug  7 19:16:52 2016
\begin{table}[H]
\centering
\begin{tabular}{rrrrr}
  \hline
 & intercept & interceptSignificant & slope & slopeSignificant \\ 
  \hline
betas & -0.40 & 1.00 & 0.81 & 1.00 \\ 
  annualBetas1 & -0.77 & 1.00 & 0.65 & 1.00 \\ 
  annualBetas2 & -0.19 & 1.00 & 0.89 & 1.00 \\ 
  annualBetas3 & -0.44 & 1.00 & 0.80 & 1.00 \\ 
  annualBetas4 & -0.14 & 1.00 & 0.91 & 1.00 \\ 
  annualBetas5 & -0.39 & 1.00 & 0.82 & 1.00 \\ 
  seasonalBetas1 & -1.02 & 1.00 & 0.37 & 1.00 \\ 
  seasonalBetas2 & -0.42 & 1.00 & 0.81 & 1.00 \\ 
  seasonalBetas3 & -1.13 & 1.00 & 0.53 & 1.00 \\ 
  seasonalBetas4 & -0.62 & 1.00 & 0.68 & 1.00 \\ 
  flowBetas1 & -0.26 & 1.00 & 0.87 & 1.00 \\ 
  flowBetas2 & -0.24 & 1.00 & 0.88 & 1.00 \\ 
  flowBetas3 & -0.49 & 1.00 & 0.77 & 1.00 \\ 
  flowBetas4 & -0.49 & 1.00 & 0.78 & 1.00 \\ 
   \hline
\end{tabular}
\caption{C3 no23} 
\end{table}
% latex table generated in R 3.2.3 by xtable 1.8-2 package
% Sun Aug  7 19:16:52 2016
\begin{table}[H]
\centering
\begin{tabular}{rrrrr}
  \hline
 & intercept & interceptSignificant & slope & slopeSignificant \\ 
  \hline
betas & -0.05 & 1.00 & 0.91 & 1.00 \\ 
  annualBetas1 & -0.05 & 1.00 & 0.83 & 1.00 \\ 
  annualBetas2 & -0.14 & 1.00 & 0.87 & 1.00 \\ 
  annualBetas3 & 0.06 & 0.00 & 0.81 & 1.00 \\ 
  annualBetas4 & -0.05 & 0.00 & 0.95 & 0.00 \\ 
  annualBetas5 & -0.06 & 1.00 & 0.94 & 0.00 \\ 
  seasonalBetas1 & 0.05 & 0.00 & 0.85 & 1.00 \\ 
  seasonalBetas2 & -0.11 & 1.00 & 0.89 & 1.00 \\ 
  seasonalBetas3 & -0.07 & 1.00 & 0.82 & 1.00 \\ 
  seasonalBetas4 & -0.01 & 0.00 & 0.87 & 1.00 \\ 
  flowBetas1 & 0.05 & 1.00 & 0.86 & 1.00 \\ 
  flowBetas2 & -0.07 & 1.00 & 0.94 & 0.00 \\ 
  flowBetas3 & -0.17 & 1.00 & 1.00 & 0.00 \\ 
  flowBetas4 & -0.04 & 0.00 & 0.94 & 0.00 \\ 
   \hline
\end{tabular}
\caption{P8 din} 
\end{table}
% latex table generated in R 3.2.3 by xtable 1.8-2 package
% Sun Aug  7 19:16:52 2016
\begin{table}[H]
\centering
\begin{tabular}{rrrrr}
  \hline
 & intercept & interceptSignificant & slope & slopeSignificant \\ 
  \hline
betas & -0.26 & 1.00 & 0.92 & 1.00 \\ 
  annualBetas1 & -0.42 & 1.00 & 0.86 & 1.00 \\ 
  annualBetas2 & -0.17 & 1.00 & 0.97 & 0.00 \\ 
  annualBetas3 & -0.09 & 0.00 & 0.97 & 0.00 \\ 
  annualBetas4 & -0.20 & 1.00 & 0.98 & 0.00 \\ 
  annualBetas5 & -0.62 & 1.00 & 0.79 & 1.00 \\ 
  seasonalBetas1 & -0.25 & 1.00 & 0.83 & 1.00 \\ 
  seasonalBetas2 & -0.55 & 1.00 & 0.84 & 1.00 \\ 
  seasonalBetas3 & -0.98 & 1.00 & 0.67 & 1.00 \\ 
  seasonalBetas4 & -0.27 & 1.00 & 0.90 & 1.00 \\ 
  flowBetas1 & -0.11 & 0.00 & 0.94 & 1.00 \\ 
  flowBetas2 & -0.20 & 1.00 & 0.98 & 0.00 \\ 
  flowBetas3 & -0.29 & 1.00 & 0.99 & 0.00 \\ 
  flowBetas4 & -0.06 & 0.00 & 0.99 & 0.00 \\ 
   \hline
\end{tabular}
\caption{P8 nh} 
\end{table}
% latex table generated in R 3.2.3 by xtable 1.8-2 package
% Sun Aug  7 19:16:52 2016
\begin{table}[H]
\centering
\begin{tabular}{rrrrr}
  \hline
 & intercept & interceptSignificant & slope & slopeSignificant \\ 
  \hline
betas & -0.06 & 1.00 & 0.91 & 1.00 \\ 
  annualBetas1 & -0.07 & 1.00 & 0.84 & 1.00 \\ 
  annualBetas2 & -0.17 & 1.00 & 0.84 & 1.00 \\ 
  annualBetas3 & 0.04 & 0.00 & 0.77 & 1.00 \\ 
  annualBetas4 & -0.06 & 1.00 & 0.95 & 0.00 \\ 
  annualBetas5 & -0.06 & 1.00 & 0.96 & 0.00 \\ 
  seasonalBetas1 & 0.02 & 0.00 & 0.87 & 1.00 \\ 
  seasonalBetas2 & -0.12 & 1.00 & 0.89 & 1.00 \\ 
  seasonalBetas3 & -0.08 & 1.00 & 0.85 & 1.00 \\ 
  seasonalBetas4 & -0.03 & 0.00 & 0.87 & 1.00 \\ 
  flowBetas1 & 0.03 & 1.00 & 0.84 & 1.00 \\ 
  flowBetas2 & -0.05 & 1.00 & 0.92 & 0.00 \\ 
  flowBetas3 & -0.13 & 1.00 & 0.91 & 0.00 \\ 
  flowBetas4 & -0.08 & 1.00 & 0.90 & 0.00 \\ 
   \hline
\end{tabular}
\caption{P8 no23} 
\end{table}
% latex table generated in R 3.2.3 by xtable 1.8-2 package
% Sun Aug  7 19:16:52 2016
\begin{table}[H]
\centering
\begin{tabular}{rrrrr}
  \hline
 & intercept & interceptSignificant & slope & slopeSignificant \\ 
  \hline
betas & -0.19 & 1.00 & 0.81 & 1.00 \\ 
  annualBetas1 & -0.32 & 1.00 & 0.70 & 1.00 \\ 
  annualBetas2 & -0.05 & 0.00 & 0.87 & 1.00 \\ 
  annualBetas3 & -0.17 & 1.00 & 0.85 & 1.00 \\ 
  annualBetas4 &  &  &  &  \\ 
  annualBetas5 & -0.17 & 1.00 & 0.86 & 1.00 \\ 
  seasonalBetas1 & -0.39 & 1.00 & 0.19 & 1.00 \\ 
  seasonalBetas2 & -0.40 & 1.00 & 0.60 & 1.00 \\ 
  seasonalBetas3 & -1.05 & 1.00 & 0.34 & 1.00 \\ 
  seasonalBetas4 & -0.21 & 1.00 & 0.80 & 1.00 \\ 
  flowBetas1 & -0.21 & 1.00 & 0.82 & 1.00 \\ 
  flowBetas2 & -0.20 & 1.00 & 0.77 & 1.00 \\ 
  flowBetas3 & -0.16 & 1.00 & 0.85 & 1.00 \\ 
  flowBetas4 & -0.20 & 1.00 & 0.78 & 1.00 \\ 
   \hline
\end{tabular}
\caption{D19 din} 
\end{table}
% latex table generated in R 3.2.3 by xtable 1.8-2 package
% Sun Aug  7 19:16:52 2016
\begin{table}[H]
\centering
\begin{tabular}{rrrrr}
  \hline
 & intercept & interceptSignificant & slope & slopeSignificant \\ 
  \hline
betas & -0.12 & 1.00 & 0.97 & 1.00 \\ 
  annualBetas1 & -0.35 & 1.00 & 0.89 & 1.00 \\ 
  annualBetas2 & 0.00 & 0.00 & 0.99 & 0.00 \\ 
  annualBetas3 & 0.01 & 0.00 & 1.02 & 0.00 \\ 
  annualBetas4 &  &  &  &  \\ 
  annualBetas5 & -0.22 & 1.00 & 0.94 & 1.00 \\ 
  seasonalBetas1 & -0.44 & 1.00 & 0.81 & 1.00 \\ 
  seasonalBetas2 & 0.24 & 0.00 & 1.07 & 0.00 \\ 
  seasonalBetas3 & -1.10 & 1.00 & 0.73 & 1.00 \\ 
  seasonalBetas4 & -0.72 & 1.00 & 0.75 & 1.00 \\ 
  flowBetas1 & -0.14 & 0.00 & 0.97 & 0.00 \\ 
  flowBetas2 & -0.14 & 0.00 & 0.95 & 0.00 \\ 
  flowBetas3 & -0.14 & 0.00 & 0.97 & 0.00 \\ 
  flowBetas4 & -0.07 & 0.00 & 0.99 & 0.00 \\ 
   \hline
\end{tabular}
\caption{D19 nh} 
\end{table}
% latex table generated in R 3.2.3 by xtable 1.8-2 package
% Sun Aug  7 19:16:52 2016
\begin{table}[H]
\centering
\begin{tabular}{rrrrr}
  \hline
 & intercept & interceptSignificant & slope & slopeSignificant \\ 
  \hline
betas & -0.26 & 1.00 & 0.78 & 1.00 \\ 
  annualBetas1 & -0.43 & 1.00 & 0.64 & 1.00 \\ 
  annualBetas2 & -0.06 & 0.00 & 0.87 & 1.00 \\ 
  annualBetas3 & -0.14 & 1.00 & 0.92 & 0.00 \\ 
  annualBetas4 &  &  &  &  \\ 
  annualBetas5 & -0.22 & 1.00 & 0.84 & 1.00 \\ 
  seasonalBetas1 & -0.55 & 1.00 & 0.18 & 1.00 \\ 
  seasonalBetas2 & -0.42 & 1.00 & 0.62 & 1.00 \\ 
  seasonalBetas3 & -1.12 & 1.00 & 0.35 & 1.00 \\ 
  seasonalBetas4 & -0.25 & 1.00 & 0.79 & 1.00 \\ 
  flowBetas1 & -0.26 & 1.00 & 0.79 & 1.00 \\ 
  flowBetas2 & -0.31 & 1.00 & 0.71 & 1.00 \\ 
  flowBetas3 & -0.17 & 1.00 & 0.85 & 1.00 \\ 
  flowBetas4 & -0.27 & 1.00 & 0.75 & 1.00 \\ 
   \hline
\end{tabular}
\caption{D19 no23} 
\end{table}
% latex table generated in R 3.2.3 by xtable 1.8-2 package
% Sun Aug  7 19:16:52 2016
\begin{table}[H]
\centering
\begin{tabular}{rrrrr}
  \hline
 & intercept & interceptSignificant & slope & slopeSignificant \\ 
  \hline
betas & -0.14 & 1.00 & 0.85 & 1.00 \\ 
  annualBetas1 & -0.25 & 1.00 & 0.74 & 1.00 \\ 
  annualBetas2 & -0.02 & 0.00 & 0.90 & 1.00 \\ 
  annualBetas3 & -0.09 & 1.00 & 0.91 & 1.00 \\ 
  annualBetas4 & -0.13 & 1.00 & 0.86 & 1.00 \\ 
  annualBetas5 & -0.12 & 1.00 & 0.92 & 1.00 \\ 
  seasonalBetas1 & -0.39 & 1.00 & 0.30 & 1.00 \\ 
  seasonalBetas2 & -0.19 & 1.00 & 0.80 & 1.00 \\ 
  seasonalBetas3 & -0.67 & 1.00 & 0.50 & 1.00 \\ 
  seasonalBetas4 & -0.18 & 1.00 & 0.80 & 1.00 \\ 
  flowBetas1 & -0.10 & 1.00 & 0.92 & 0.00 \\ 
  flowBetas2 & -0.09 & 1.00 & 0.87 & 1.00 \\ 
  flowBetas3 & -0.17 & 1.00 & 0.84 & 1.00 \\ 
  flowBetas4 & -0.16 & 1.00 & 0.82 & 1.00 \\ 
   \hline
\end{tabular}
\caption{D26 din} 
\end{table}
% latex table generated in R 3.2.3 by xtable 1.8-2 package
% Sun Aug  7 19:16:52 2016
\begin{table}[H]
\centering
\begin{tabular}{rrrrr}
  \hline
 & intercept & interceptSignificant & slope & slopeSignificant \\ 
  \hline
betas & -0.36 & 1.00 & 0.85 & 1.00 \\ 
  annualBetas1 & -0.42 & 1.00 & 0.84 & 1.00 \\ 
  annualBetas2 & -0.19 & 1.00 & 0.90 & 1.00 \\ 
  annualBetas3 & -0.37 & 1.00 & 0.83 & 1.00 \\ 
  annualBetas4 & -0.42 & 1.00 & 0.85 & 1.00 \\ 
  annualBetas5 & -0.39 & 1.00 & 0.85 & 1.00 \\ 
  seasonalBetas1 & -0.75 & 1.00 & 0.65 & 1.00 \\ 
  seasonalBetas2 & -1.41 & 1.00 & 0.48 & 1.00 \\ 
  seasonalBetas3 & -0.91 & 1.00 & 0.67 & 1.00 \\ 
  seasonalBetas4 & -0.58 & 1.00 & 0.72 & 1.00 \\ 
  flowBetas1 & -0.48 & 1.00 & 0.80 & 1.00 \\ 
  flowBetas2 & -0.33 & 1.00 & 0.88 & 1.00 \\ 
  flowBetas3 & -0.40 & 1.00 & 0.84 & 1.00 \\ 
  flowBetas4 & -0.18 & 0.00 & 0.92 & 0.00 \\ 
   \hline
\end{tabular}
\caption{D26 nh} 
\end{table}
% latex table generated in R 3.2.3 by xtable 1.8-2 package
% Sun Aug  7 19:16:52 2016
\begin{table}[H]
\centering
\begin{tabular}{rrrrr}
  \hline
 & intercept & interceptSignificant & slope & slopeSignificant \\ 
  \hline
betas & -0.19 & 1.00 & 0.84 & 1.00 \\ 
  annualBetas1 & -0.37 & 1.00 & 0.69 & 1.00 \\ 
  annualBetas2 & -0.06 & 0.00 & 0.88 & 1.00 \\ 
  annualBetas3 & -0.12 & 1.00 & 0.92 & 1.00 \\ 
  annualBetas4 & -0.13 & 1.00 & 0.89 & 1.00 \\ 
  annualBetas5 & -0.14 & 1.00 & 0.92 & 1.00 \\ 
  seasonalBetas1 & -0.62 & 1.00 & 0.22 & 1.00 \\ 
  seasonalBetas2 & -0.21 & 1.00 & 0.81 & 1.00 \\ 
  seasonalBetas3 & -0.82 & 1.00 & 0.50 & 1.00 \\ 
  seasonalBetas4 & -0.26 & 1.00 & 0.80 & 1.00 \\ 
  flowBetas1 & -0.10 & 0.00 & 0.95 & 0.00 \\ 
  flowBetas2 & -0.10 & 0.00 & 0.88 & 1.00 \\ 
  flowBetas3 & -0.21 & 1.00 & 0.84 & 1.00 \\ 
  flowBetas4 & -0.24 & 1.00 & 0.80 & 1.00 \\ 
   \hline
\end{tabular}
\caption{D26 no23} 
\end{table}
% latex table generated in R 3.2.3 by xtable 1.8-2 package
% Sun Aug  7 19:16:52 2016
\begin{table}[H]
\centering
\begin{tabular}{rrrrr}
  \hline
 & intercept & interceptSignificant & slope & slopeSignificant \\ 
  \hline
betas & -0.15 & 1.00 & 0.87 & 1.00 \\ 
  annualBetas1 & -0.32 & 1.00 & 0.67 & 1.00 \\ 
  annualBetas2 & -0.06 & 0.00 & 0.87 & 1.00 \\ 
  annualBetas3 & -0.16 & 1.00 & 0.86 & 1.00 \\ 
  annualBetas4 & -0.12 & 1.00 & 0.91 & 1.00 \\ 
  annualBetas5 & -0.11 & 1.00 & 0.93 & 1.00 \\ 
  seasonalBetas1 & -0.28 & 1.00 & 0.24 & 1.00 \\ 
  seasonalBetas2 & -0.28 & 1.00 & 0.71 & 1.00 \\ 
  seasonalBetas3 & -0.72 & 1.00 & 0.58 & 1.00 \\ 
  seasonalBetas4 & -0.26 & 1.00 & 0.82 & 1.00 \\ 
  flowBetas1 & -0.19 & 1.00 & 0.86 & 1.00 \\ 
  flowBetas2 & -0.12 & 0.00 & 0.85 & 1.00 \\ 
  flowBetas3 & -0.13 & 1.00 & 0.89 & 1.00 \\ 
  flowBetas4 & -0.14 & 1.00 & 0.85 & 1.00 \\ 
   \hline
\end{tabular}
\caption{D28 din} 
\end{table}
% latex table generated in R 3.2.3 by xtable 1.8-2 package
% Sun Aug  7 19:16:52 2016
\begin{table}[H]
\centering
\begin{tabular}{rrrrr}
  \hline
 & intercept & interceptSignificant & slope & slopeSignificant \\ 
  \hline
betas & -0.42 & 1.00 & 0.88 & 1.00 \\ 
  annualBetas1 & -0.90 & 1.00 & 0.73 & 1.00 \\ 
  annualBetas2 & -0.20 & 0.00 & 0.94 & 0.00 \\ 
  annualBetas3 & -0.33 & 1.00 & 0.92 & 1.00 \\ 
  annualBetas4 & -0.41 & 1.00 & 0.90 & 1.00 \\ 
  annualBetas5 & -0.47 & 1.00 & 0.87 & 1.00 \\ 
  seasonalBetas1 & -1.24 & 1.00 & 0.52 & 1.00 \\ 
  seasonalBetas2 & -1.03 & 1.00 & 0.71 & 1.00 \\ 
  seasonalBetas3 & -2.27 & 1.00 & 0.46 & 1.00 \\ 
  seasonalBetas4 & -0.87 & 1.00 & 0.74 & 1.00 \\ 
  flowBetas1 & -0.53 & 1.00 & 0.87 & 1.00 \\ 
  flowBetas2 & -0.32 & 1.00 & 0.90 & 1.00 \\ 
  flowBetas3 & -0.31 & 1.00 & 0.91 & 1.00 \\ 
  flowBetas4 & -0.54 & 1.00 & 0.84 & 1.00 \\ 
   \hline
\end{tabular}
\caption{D28 nh} 
\end{table}
% latex table generated in R 3.2.3 by xtable 1.8-2 package
% Sun Aug  7 19:16:52 2016
\begin{table}[H]
\centering
\begin{tabular}{rrrrr}
  \hline
 & intercept & interceptSignificant & slope & slopeSignificant \\ 
  \hline
betas & -0.17 & 1.00 & 0.86 & 1.00 \\ 
  annualBetas1 & -0.37 & 1.00 & 0.67 & 1.00 \\ 
  annualBetas2 & -0.07 & 0.00 & 0.86 & 1.00 \\ 
  annualBetas3 & -0.16 & 1.00 & 0.88 & 1.00 \\ 
  annualBetas4 & -0.12 & 1.00 & 0.92 & 1.00 \\ 
  annualBetas5 & -0.13 & 1.00 & 0.93 & 1.00 \\ 
  seasonalBetas1 & -0.39 & 1.00 & 0.23 & 1.00 \\ 
  seasonalBetas2 & -0.26 & 1.00 & 0.75 & 1.00 \\ 
  seasonalBetas3 & -0.74 & 1.00 & 0.60 & 1.00 \\ 
  seasonalBetas4 & -0.28 & 1.00 & 0.81 & 1.00 \\ 
  flowBetas1 & -0.21 & 1.00 & 0.86 & 1.00 \\ 
  flowBetas2 & -0.13 & 1.00 & 0.85 & 1.00 \\ 
  flowBetas3 & -0.15 & 1.00 & 0.89 & 1.00 \\ 
  flowBetas4 & -0.18 & 1.00 & 0.83 & 1.00 \\ 
   \hline
\end{tabular}
\caption{D28 no23} 
\end{table}


\section{Observations}

\begin{itemize}
\item both methods have predictions (visually) very close to one another
\item RMSE for GAM less than that of WRTDS and same for deviance in all categories 
\item for the flow-normalized trends, the average for GAMs is often greater than or equal to (actually a lot of ties) that of WRTDS, the other way around for annual2, flows 2 and 3 sometimes 
%\item \textcolor{red}{fixed percent change, need to comment now}
\item Negative PC: C10din, C10no23, C3 din, C3nh, C3no23, P8no23, D26nh
\item Positive PC: C10nh, P8din (big discrepancy between methods), P8nh, D19din, D19nh, D19no23, D26din, D26no23, D28din, D28nh, D28no23
\item  Note: for  AvgDiff negative percentages indicate WRTDS predictions were lower than GAM predictions
\item Mostly Neg: C10din, C10no23, P8din, P8no23
\item Mostly Pos: C10nh, C3din, C3nh, C3no23, P8nh, D19nh, D19no23, D26nh, D28nh
\item About 50/50: D19din, D26din, D26no23, D28din, D28no23
\item RMSE all pretty reasonable overall, breakdown reasonably uniform (don't want to read too much into this until full data is available in case missingness is driving anything)
\item Note: intercept > 0 WRTDS > GAM on average, slope < 1 GAM fits wider range
%\item \textcolor{red}{fixed significantly different than 1, need to comment}
\item all have slopes significantly different than 1 overall
%\item C10din, P8din, P8no23, D19din, D28din, percent changes have different signs between the methods among all breakdowns which seems weird
%\item  D19nh opposite sign overall, but match on all breakdowns
%\item D19no23, D26din, D26no23, D28nh (don't match seasonal1 either), D28no23, opposite except for all flow, and seasonal 2-4
%\item D26nh opposite signs in annual breakdowns
\item intercept significance should put the results from AvgDiff in perspective
\item Overall significant (so most): C10din, C10no23, C3din, C3nh, C3no23, P8din, P8nh, P8no23, D19din, D19nh, D19no23, D26din, D26nh, D26no23, D28din, D28nh, D28no23
\end{itemize}

\subsection{Observations on Results of Different Parameter Experiment }
\begin{itemize}
\item Stations of interest: D7, D19 chosen to be fairly representative (in the center portion)
\item Expand k: visually the predictions look similar to those of the GAM model with the original choices of k
\item WRTDS predictions are less extreme than the GAM predictions: so if GAM is always ``winning" it means it gets high/low enough in the right places and isn't losing big by being large in magnitude when it shouldn't
\item expanded k version always has smaller RMSE breakdown than the original GAM model which means it still has smaller RMSE than WRTDS
\item can't expand k any further because of identifiability constraints
\item Original window sizes (0.5, 10, 0.5) try smaller window sizes (0.25, 5, 0.25) and (0.1, 2, 0.1) and bigger window sizes (0.75, 15, 0.75) and (1,20,1)
\item Still overall the expanded GAM does better than all of the WRTDS models
\item of the WRTDS models, the smaller window model generally does better, with a few breakdown categories having the original WRTDS model being the best
\end{itemize}

\end{document}